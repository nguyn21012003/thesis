\documentclass[10Pt]{article}
\usepackage[utf8]{vietnam}
\usepackage{xcolor}
\usepackage[utf8]{inputenc}
\usepackage{fontsize}
\changefontsize[15pt]{15pt}
\usepackage{commath}
\usepackage{blindtext}
\usepackage{xcolor}
\usepackage{amssymb}
\usepackage{slashed}
\usepackage{indentfirst,parskip}
\setlength{\parindent}{2em}
\usepackage{pdfpages}
\usepackage{graphicx}
%\usepackage{tikz-feynman}
\usepackage{nccmath}
\usepackage{mathtools}
\usepackage{amsfonts}
\usepackage{amsmath,systeme}
\usepackage[thinc]{esdiff}
\usepackage{hyperref}
\usepackage{dirtytalk,bm,physics}
\usepackage{tikz}
\usepackage{lipsum}
\usepackage{fancyhdr}
%footnote
\pagestyle{fancy}
\renewcommand{\headrulewidth}{0pt}%
\fancyhf{}%
\fancyfoot[LE,LO]{Vật lý Lý thuyết}%
\fancyfoot[C]{\hspace{4cm} \thepage}%

\usetikzlibrary{shapes.geometric, arrows}

\usepackage{geometry}
\geometry{
	a4paper,
	total={170mm,257mm},
	left=20mm,
	top=20mm,
}

\renewcommand{\baselinestretch}{2.0}
\usetikzlibrary{arrows.spaced}
\usetikzlibrary{animations,quotes}
%gian do
\tikzstyle{startstop} = [rectangle, rounded corners, minimum width=3cm, minimum height=1cm, text centered,draw=black, fill=white!30]
\tikzstyle{arrow} = [thick,->,>=stealth]

\title{\Huge{HAMILTONIAN}}

\hypersetup{
	colorlinks=true,
	linkcolor=black,
	filecolor=magenta,      
	urlcolor=cyan,
	pdftitle={LTTHD},
	pdfpagemode=FullScreen,
}

\urlstyle{same}


\definecolor{mycolor}{RGB}{255,0,0}

\begin{document}
	\author{DAO DUY TUNG}
	
	\maketitle
	
	\textbf{**h11}
\begin{align*}
	H^{22}_{11} &= \sum_{\mathbf{R}} e^{i \mathbf{k} \cdot \mathbf{R}} E^{22}_{11}(\mathbf{R}) 
	= e^{i \mathbf{k} \cdot \mathbf{R}_1} E^{22}_{11}(\mathbf{R}_1)
	+ e^{i \mathbf{k} \cdot \mathbf{R}_2} E^{22}_{11}(\mathbf{R}_2)
	+ e^{i \mathbf{k} \cdot \mathbf{R}_3} E^{22}_{11}(\mathbf{R}_3)\\
	&+ e^{i \mathbf{k} \cdot \mathbf{R}_4} E^{22}_{11}(\mathbf{R}_4)
	+ e^{i \mathbf{k} \cdot \mathbf{R}_5} E^{22}_{11}(\mathbf{R}_5)
	+ e^{i \mathbf{k} \cdot \mathbf{R}_6} E^{22}_{11}(\mathbf{R}_6)
	+ E^{22}_{11}(\mathbf{0})\\
	&= e^{i k_x a} E^{22}_{11}(\mathbf{R}_1)
	+ e^{i \left( k_x \frac{a}{2} - k_y \frac{a\sqrt{3}}{2} \right)} E^{22}_{11}(\mathbf{R}_2)
	+ e^{i \left( -k_x \frac{a}{2} - k_y \frac{a\sqrt{3}}{2} \right)} E^{22}_{11}(\mathbf{R}_3)\\
	&+ e^{-i k_x a } E^{22}_{11}(\mathbf{R}_4) 
	+ e^{i \left( -k_x \frac{a}{2} + k_y \frac{a\sqrt{3}}{2} \right)} E^{22}_{11}(\mathbf{R}_5)
	+ e^{i \left( k_x \frac{a}{2} + k_y \frac{a\sqrt{3}}{2} \right)} E^{22}_{11}(\mathbf{R}_6) + \epsilon_2\\
	&= e^{2i\alpha} t_{11} + e^{i \left( \alpha - \beta \right)} \frac{t_{11} -\sqrt{3}t_{12} - \sqrt{3}c_{21} + 3t_{22}}{4}\\
	&+ e^{i \left( -\alpha - \beta \right)} \frac{t_{11} -\sqrt{3}t_{12} - \sqrt{3}c_{21} + 3t_{22}}{4} + e^{-2i\alpha} t_{11}\\
	&+ e^{i \left( -\alpha + \beta \right)} \frac{t_{11} +\sqrt{3}t_{12} + \sqrt{3}c_{21} + 3t_{22}}{4} + e^{i \left( \alpha + \beta \right)} \frac{t_{11} +\sqrt{3}t_{12} + \sqrt{3}c_{21} + 3t_{22}}{4} + \epsilon_2\\
	&= 2t_{11}cos(2\alpha) + \left (t_{11} + 3t_{22}\right )cos(\alpha)cos(\beta) + \epsilon_2\\ 
	&\textbf{Lưu ý ở đây đã sử dụng tính chất Hermite của $h_{11}$ phải là số thực}\\ &\Rightarrow t_{12} = - t_{21}
\end{align*}

\begin{align*}
	E^{22} (\mathbf{R_2}) &= E^{22}(\sigma_\nu^{''} \mathbf{R_1}) = D^2(\sigma^{''}_\nu) E^{22} (\mathbf{R_1}) [D^2(\sigma_\nu ^{''})]^\dagger\\
	&= 
	\begin{bmatrix}
		\frac{1}{2} & -\frac{\sqrt{3}}{2} \\
		-\frac{\sqrt{3}}{2} & -\frac{1}{2}
	\end{bmatrix} \begin{bmatrix}
	a & b \\
	c & d
	\end{bmatrix}\begin{bmatrix}
	\frac{1}{2} & -\frac{\sqrt{3}}{2} \\
	-\frac{\sqrt{3}}{2} & -\frac{1}{2}
	\end{bmatrix}\text{Trong đó}\begin{bmatrix}
	a = t_{11} \\
	b = t_{12} \\
	c = c_{21} \\
	d = t_{22}
	\end{bmatrix}\\
	&\Rightarrow E^{22}_{11} (\mathbf{R_2}) = \frac{a - \sqrt{3}b - \sqrt{3}c + 3d}{4}
\end{align*}
Tương tự ta tìm được:
\begin{align*}
	 E^{22}_{11} (\mathbf{R_3}) &= \frac{a - \sqrt{3}b - \sqrt{3}c + 3d}{4} \\
	 E^{22}_{11} (\mathbf{R_4}) &= a \\
	 E^{22}_{11} (\mathbf{R_5}) &= \frac{a + \sqrt{3}b + \sqrt{3}c + 3d}{4} \\
	 E^{22}_{11} (\mathbf{R_6}) &= \frac{a + \sqrt{3}b + \sqrt{3}c + 3d}{4} \\
\end{align*}

	\textbf{**h12}
\begin{align*}
	H^{22}_{12} &= \sum_{\mathbf{R}} e^{i \mathbf{k} \cdot \mathbf{R}} E^{22}_{12}(\mathbf{R}) \\
	&= e^{i \mathbf{k} \cdot \mathbf{R}_1} E^{22}_{12}(\mathbf{R}_1)
	+ e^{i \mathbf{k} \cdot \mathbf{R}_2} E^{22}_{12}(\mathbf{R}_2)
	+ e^{i \mathbf{k} \cdot \mathbf{R}_3} E^{22}_{12}(\mathbf{R}_3) \\
	&\quad + e^{i \mathbf{k} \cdot \mathbf{R}_4} E^{22}_{12}(\mathbf{R}_4)
	+ e^{i \mathbf{k} \cdot \mathbf{R}_5} E^{22}_{12}(\mathbf{R}_5)
	+ e^{i \mathbf{k} \cdot \mathbf{R}_6} E^{22}_{12}(\mathbf{R}_6)
	+ E^{22}_{12}(\mathbf{0}) \\
	&= e^{i k_x a} E^{22}_{12}(\mathbf{R}_1)
	+ e^{i \left( k_x \frac{a}{2} - k_y \frac{a\sqrt{3}}{2} \right)} E^{22}_{12}(\mathbf{R}_2) \\
	&\quad + e^{i \left( -k_x \frac{a}{2} - k_y \frac{a\sqrt{3}}{2} \right)} E^{22}_{12}(\mathbf{R}_3) \\
	&\quad + e^{-i k_x a } E^{22}_{12}(\mathbf{R}_4) 
	+ e^{i \left( -k_x \frac{a}{2} + k_y \frac{a\sqrt{3}}{2} \right)} E^{22}_{12}(\mathbf{R}_5) \\
	&\quad + e^{i \left( k_x \frac{a}{2} + k_y \frac{a\sqrt{3}}{2} \right)} E^{22}_{12}(\mathbf{R}_6) \\
	&= e^{2i\alpha} t_{12} 
	+ e^{i \left( \alpha - \beta \right)} \frac{-\sqrt{3}t_{11} - t_{12} + 3c_{21} + \sqrt{3}t_{22}}{4} \\
	&\quad + e^{i \left( -\alpha - \beta \right)} \frac{\sqrt{3}t_{11} + t_{12} - 3c_{21} - \sqrt{3}t_{22}}{4} \\
	&\quad - e^{-2i\alpha} t_{12} 
	+ e^{i \left( -\alpha + \beta \right)} \frac{-\sqrt{3}t_{11} + t_{12} - 3c_{21} + \sqrt{3}t_{22}}{4} \\
	&\quad + e^{i \left( \alpha + \beta \right)} \frac{\sqrt{3}t_{11} - t_{12} + 3c_{21} - \sqrt{3}t_{22}}{4} \\
	&= \sqrt{3}(t_{22} - t_{11}) \sin(\alpha) \sin(\beta) 
	+ 4i t_{12} \sin(\alpha) (\cos(\alpha) - \cos(\beta)) + 3i c_{21} sin(\alpha)cos(\beta) 
\end{align*}

\begin{align*}
	E^{22} (\mathbf{R_2}) &= E^{22}(\sigma_\nu^{''} \mathbf{R_1}) = D^2(\sigma^{''}_\nu) E^{22} (\mathbf{R_1}) [D^2(\sigma_\nu ^{''})]^\dagger\\
	&= 
	\begin{bmatrix}
		\frac{1}{2} & -\frac{\sqrt{3}}{2} \\
		-\frac{\sqrt{3}}{2} & -\frac{1}{2}
	\end{bmatrix} \begin{bmatrix}
		a & b \\
		c & d
	\end{bmatrix}\begin{bmatrix}
		\frac{1}{2} & -\frac{\sqrt{3}}{2} \\
		-\frac{\sqrt{3}}{2} & -\frac{1}{2}
	\end{bmatrix}\text{Trong đó}\begin{bmatrix}
		a = t_{11} \\
		b = t_{12} \\
		c = c_{21} \\
		d = t_{22}
	\end{bmatrix}\\
	&\Rightarrow E^{22}_{12} (\mathbf{R_2}) = \frac{-\sqrt{3}a - b + 3c + \sqrt{3}d}{4}
\end{align*}
Tương tự ta tìm được:
\begin{align*}
	E^{22}_{12} (\mathbf{R_3}) &= \frac{\sqrt{3}a + b - 3c - \sqrt{3}d}{4} \\
	E^{22}_{12} (\mathbf{R_4}) &= -b \\
	E^{22}_{12} (\mathbf{R_5}) &= \frac{\sqrt{3}a + b - 3c + \sqrt{3}d}{4} \\
	E^{22}_{12} (\mathbf{R_6}) &= \frac{\sqrt{3}a - b + 3c - \sqrt{3}d}{4} \\
\end{align*}

\end{document}