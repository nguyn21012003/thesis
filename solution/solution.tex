\documentclass{report}
\usepackage[utf8]{vietnam}
\usepackage[utf8]{inputenc}
\usepackage{anyfontsize,fontsize}
\changefontsize[13pt]{13pt}
\usepackage{commath}
\usepackage{blindtext}
\usepackage{parskip}
\usepackage{xcolor}
\usepackage{amssymb}
\usepackage{slashed,cancel}
\usepackage{indentfirst}
\usepackage{pdfpages}
\usepackage{graphicx}
\usepackage{upgreek}
\usepackage{nccmath}
\usepackage{mathtools}
\usepackage{amsfonts}
\usepackage{amsmath,systeme,bbold}
\usepackage[thinc]{esdiff}
\usepackage{hyperref}
\usepackage{dirtytalk,bm,physics,upgreek}
\usepackage{lipsum}
\usepackage{fancyhdr}
%footnote
\pagestyle{fancy}
\renewcommand{\headrulewidth}{0pt}%
\fancyhf{}%
\fancyfoot[L]{Vật lý Lý thuyết}%
\fancyfoot[C]{\hspace{6.5cm} \thepage}%


\usepackage{geometry}
\geometry{
    a4paper,
    total = {190mm,257mm},
    top = 20mm,
    left = 10mm
}

\renewcommand{\baselinestretch}{2.0}


\newcommand{\f}[2]{\dfrac{#1}{#2}}

\DeclareRobustCommand{\rchi}{{\mathpalette\irchi\relax}}
\newcommand{\irchi}[2]{\raisebox{\depth}{$#1\chi$}}




\title{\Huge{Thesis}}

\hypersetup{
    colorlinks=true,
    linkcolor=red,
    filecolor=magenta,
    urlcolor=cyan,
    pdftitle={Thesis},
    pdfpagemode=FullScreen,
}

\urlstyle{same}

\begin{document}
\setlength{\parindent}{20pt}
\newpage
\author{TRẦN KHÔI NGUYÊN \\ VẬT LÝ LÝ THUYẾT}
\maketitle

\clearpage


Từ Hamiltonian $H^{jj'}_{\mu\mu'} \left( \mathbf{k}\right) = \sum_{\mathbf{R}}
	e^{i \mathbf{k\cdot R}} E^{jj'}_{\mu\mu'} \left(\textbf{R}\right) $ trong đó
\begin{align*}
	E^{jj'}_{\mu\mu'} \left(\textbf{R}\right) = \bra*{\phi^j_\mu \left(\textbf{r}\right)}\hat{H}\ket*{\phi^{j'}_{\mu'}\left(\textbf{r} - \textbf{R}\right)}
\end{align*}

\begin{align*}
	\ket*{\phi^1_1} = d_{z^2} , \quad \ket*{\phi^2_1} = d_{xy} , \quad \ket*{\phi^2_2} = d_{x^2 - y^2}
\end{align*}

\begin{align*}
	H^{jj'}_{\mu\mu'} \left( \mathbf{k}\right) & =  \; \sum_{\mu\mu' jj'} e^{i \mathbf{k\cdot R_1}} E^{jj'}_{\mu\mu'} \left(\mathbf{R_1}\right)
	+ \sum_{\mu\mu' jj'} e^{i \mathbf{k\cdot R_2}} E^{jj'}_{\mu\mu'} \left(\mathbf{R_2}\right)
	+ \sum_{\mu\mu' jj'} e^{i \mathbf{k\cdot R_3}} E^{jj'}_{\mu\mu'} \left(\mathbf{R_3}\right)                                                  \\
	                                           & + \sum_{\mu\mu' jj'} e^{i \mathbf{k\cdot R_4}} E^{jj'}_{\mu\mu'} \left(\mathbf{R_4}\right)
	+ \sum_{\mu\mu' jj'} e^{i \mathbf{k\cdot R_5}} E^{jj'}_{\mu\mu'} \left(\mathbf{R_5}\right)
	+ \sum_{\mu\mu' jj'} e^{i \mathbf{k\cdot R_6}} E^{jj'}_{\mu\mu'} \left(\mathbf{R_6}\right)
\end{align*}

\[
	\renewcommand{\arraystretch}{0.75}
	H^{NN} = \begin{pmatrix}
		h_{0}   & h_{1}    & h_{2}  \\
		h_{1}^* & h_{11}   & h_{12} \\
		h_{2}^* & h_{12}^* & h_{22}
	\end{pmatrix}
\]

\begin{align*}
	h_0 = \sum_{R} e^{i\mathbf{k\cdot R}} \bra*{\phi_1^1 \left(\mathbf{r}\right)} H \ket*{\phi_1^1\left(\mathbf{r}-\mathbf{R}\right)}; \quad h_1 = \sum_{R} e^{i\mathbf{k\cdot R}} \bra*{\phi_1^1 \left(\mathbf{r}\right)} H \ket*{\phi_1^2\left(\mathbf{r}-\mathbf{R}\right)}     \\
	h_2 = \sum_{R} e^{i\mathbf{k\cdot R}} \bra*{\phi_1^1 \left(\mathbf{r}\right)} H \ket*{\phi_2^2\left(\mathbf{r}-\mathbf{R}\right)}; \quad  h_{11} = \sum_{R} e^{i\mathbf{k\cdot R}} \bra*{\phi_1^2 \left(\mathbf{r}\right)} H \ket*{\phi_1^2\left(\mathbf{r}-\mathbf{R}\right)} \\
	h_{12} = \sum_{R} e^{i\mathbf{k\cdot R}} \bra*{\phi_1^2 \left(\mathbf{r}\right)} H \ket*{\phi_2^2\left(\mathbf{r}-\mathbf{R}\right)}; \quad h_{22} = \sum_{R} e^{i\mathbf{k\cdot R}} \bra*{\phi_2^2 \left(\mathbf{r}\right)} H \ket*{\phi_2^2\left(\mathbf{r}-\mathbf{R}\right)}
\end{align*}

Lại có $E^{jj'} \left(\hat{g_n}\textbf{R}\right)= D^j(\hat{g_n}) E^{jj'}
	\left(\textbf{R}\right)\left[D^j(\hat{g_n})\right]^{\dagger}$

trong đó $\hat{g_n} = \{E,C_3,C^2_3,\sigma_\nu,\sigma'_\nu,\sigma''_\nu\}$

trong đó $D^1(\hat{g_n}) = 1$
\[
	\renewcommand{\arraystretch}{0.75}
	D^2(E) = \begin{pmatrix}
		1 & 0 \\
		0 & 1
	\end{pmatrix}
\]
\[
	\renewcommand{\arraystretch}{0.75}
	D^2(\hat{C_3}) = \begin{pmatrix}
		cos\phi & -sin\phi \\
		sin\phi & cos\phi
	\end{pmatrix}
	= \begin{pmatrix}
		-\frac{1}{2}       & -\frac{\sqrt{3}}{2} \\
		\frac{\sqrt{3}}{2} & -\frac{1}{2}
	\end{pmatrix}
\]
\[
	\renewcommand{\arraystretch}{0.75}
	D^2(\hat{C_3}^2) =
	\begin{pmatrix}
		-\frac{1}{2}        & \frac{\sqrt{3}}{2} \\
		\frac{-\sqrt{3}}{2} & -\frac{1}{2}
	\end{pmatrix}
\]

Để tìm được $D^2(\sigma_\nu)$ ta cố định $\bigtriangleup$ ABC : A($\dfrac{1}{2},\dfrac{\sqrt{3}}{2}$),B(1,0), C(0,0).

Khi đổi chỗ A $\leftrightarrow$ B, ta được ma trận:

\[
	\renewcommand{\arraystretch}{0.75}
	\begin{pmatrix}
		1 \\
		0
	\end{pmatrix}
	= D^2(\sigma_\nu)
	\begin{pmatrix}
		\frac{1}{2} \\
		\frac{\sqrt{3}}{2}
	\end{pmatrix}
	\Rightarrow  D^2(\sigma_\nu) =
	\begin{pmatrix}
		\frac{1}{2}        & \frac{\sqrt{3}}{2} \\
		\frac{\sqrt{3}}{2} & -\frac{1}{2}
	\end{pmatrix}
\]

Ta có $ \vec{R_5} = \sigma'_\nu \vec{R_4} \; \text{ mà }\; C_3^2 \vec{R_5} =
	\vec{R_1} \Rightarrow C_3^2 \sigma'_\nu \vec{R_4} = \vec{R_1}$

\[
	\renewcommand{\arraystretch}{0.75}
	\Rightarrow
	\begin{pmatrix}
		-\frac{1}{2}        & \frac{\sqrt{3}}{2} \\
		-\frac{\sqrt{3}}{2} & -\frac{1}{2}
	\end{pmatrix}
	\begin{pmatrix}
		a & b \\
		c & d
	\end{pmatrix}
	\begin{pmatrix}
		-1 \\
		0
	\end{pmatrix}
	=
	\begin{pmatrix}
		1 \\
		0
	\end{pmatrix}
\]

\[
	\renewcommand{\arraystretch}{0.75}
	\Rightarrow D^2\left(\sigma'_\nu\right) =
	\begin{pmatrix}
		\frac{1}{2}         & -\frac{\sqrt{3}}{2} \\
		-\frac{\sqrt{3}}{2} & -\frac{1}{2}
	\end{pmatrix}
\]
Tương tự ta tính cho
\[
	\renewcommand{\arraystretch}{0.75}
	D^2\left(\sigma''_\nu\right) =
	\begin{pmatrix}
		-1 & 0 \\
		0  & 1
	\end{pmatrix}
\]
\noindent Toán tử $C_3$ đánh lên $\textbf{R}_1$ ta được $\rightarrow$ $\textbf{R}_5$ (dưới dạng ma trận)\\
Toán tử $C_3^2$ đánh lên $\textbf{R}_1$ ta được $\rightarrow$ $\textbf{R}_3$ (dưới dạng ma trận)\\
Toán tử $\sigma_\nu$ đánh lên $\textbf{R}_1$ ta được $\rightarrow$ $\textbf{R}_6$ (dưới dạng ma trận)\\
Toán tử $\sigma'_\nu$ đánh lên $\textbf{R}_1$ ta được $\rightarrow$ $\textbf{R}_2$ (dưới dạng ma trận)\\
Toán tử $\sigma''_\nu$ đánh lên $\textbf{R}_1$ ta được $\rightarrow$ $\textbf{R}_4$ (dưới dạng ma trận)

Kiểm tra điều trên:

\[
	\renewcommand{\arraystretch}{0.75}
	D^2\left(C_3^2\right)R_1 =
	\begin{pmatrix}
		-\frac{1}{2}        & \frac{\sqrt{3}}{2} \\
		-\frac{\sqrt{3}}{2} & -\frac{1}{2}
	\end{pmatrix}
	\begin{pmatrix}
		1 \\
		0
	\end{pmatrix}
	=
	\begin{pmatrix}
		-\frac{1}{2} \\
		-\frac{\sqrt{3}}{2}
	\end{pmatrix}
	= \textbf{R}_3
\]
\[
	\renewcommand{\arraystretch}{0.75}
	D^2\left(\sigma'_\nu\right)R_1 =
	\begin{pmatrix}
		\frac{1}{2}         & -\frac{\sqrt{3}}{2} \\
		-\frac{\sqrt{3}}{2} & -\frac{1}{2}
	\end{pmatrix}
	\begin{pmatrix}
		1 \\
		0
	\end{pmatrix}
	=
	\begin{pmatrix}
		\frac{1}{2} \\
		-\frac{\sqrt{3}}{2}
	\end{pmatrix}
	= \textbf{R}_2
\]

\textbf{ $\ast$ h0}
\begin{align*}
	h_0 & = \sum_{\mathbf{R} \neq 0} e^{i\mathbf{k\cdot R}} \bra*{\phi_1^1 \left(\mathbf{r}\right)} H \ket*{\phi_1^1\left(\mathbf{r}-\mathbf{R}\right)}
	+ \bra*{\phi_1^1 \left(\mathbf{r}\right)} H \ket*{\phi_1^1\left(\mathbf{r}\right)}                                                                                                                                                                                                                                          \\
	    & = e^{i\mathbf{k\cdot R_1}} \bra*{\phi_1^1 \left(\mathbf{r}\right)} H \ket*{\phi_1^1\left(\mathbf{r}-\mathbf{R_1}\right)} + e^{i\mathbf{k\cdot R_4}} \bra*{\phi_1^1 \left(\mathbf{r}\right)} H \ket*{\phi_1^1\left(\mathbf{r}-\mathbf{R_4}\right)}                                                                     \\
	    & + e^{i\mathbf{k\cdot R_2}} \bra*{\phi_1^1 \left(\mathbf{r}\right)} H \ket*{\phi_1^1\left(\mathbf{r}-\mathbf{R_2}\right)} + e^{i\mathbf{k\cdot R_5}} \bra*{\phi_1^1 \left(\mathbf{r}\right)} H \ket*{\phi_1^1\left(\mathbf{r}-\mathbf{R_5}\right)}                                                                     \\
	    & + e^{i\mathbf{k\cdot R_3}} \bra*{\phi_1^1 \left(\mathbf{r}\right)} H \ket*{\phi_1^1\left(\mathbf{r}-\mathbf{R_3}\right)} + e^{i\mathbf{k\cdot R_6}} \bra*{\phi_1^1 \left(\mathbf{r}\right)} H \ket*{\phi_1^1\left(\mathbf{r}-\mathbf{R_6}\right)} + \epsilon_1                                                        \\
	    & = e^{i k_x a} E_{11}^{11}\left(\mathbf{R_1}\right) + e^{-i k_x a} E_{11}^{11}\left(\mathbf{R_4}\right) +  e^{i \left( k_x \frac{a}{2} - k_y \frac{a\sqrt{3}}{2} \right)} E_{11}^{11}\left(\mathbf{R_2}\right) +  e^{-i \left( k_x \frac{a}{2} - k_y \frac{a\sqrt{3}}{2} \right)} E_{11}^{11}\left(\mathbf{R_5}\right) \\
	    & + e^{-i \left( k_x \frac{a}{2} + k_y \frac{a\sqrt{3}}{2} \right)} E_{11}^{11}\left(\mathbf{R_3}\right) + e^{i \left( k_x \frac{a}{2} + k_y \frac{a\sqrt{3}}{2} \right)} E_{11}^{11}\left(\mathbf{R_6}\right) + \epsilon_1                                                                                             \\
	    & = 2 E_{11}^{11}\left(\mathbf{R_1}\right) \left(cos2\alpha + 2cos\alpha\; cos\beta\right) + \epsilon_1
\end{align*}

\noindent \textbf{ $\ast$ h1}
\begin{align*}
	h_1 & = \sum_{\mathbf{R} \neq 0} e^{i\mathbf{k\cdot R}} \bra*{\phi_1^1 \left(\mathbf{r}\right)} H \ket*{\phi_1^2\left(\mathbf{r}-\mathbf{R}\right)}                                                                                                                                                                       \\
	    & = e^{i k_x a} E_{11}^{12}\left(\mathbf{R_1}\right) + e^{-i k_x a} E_{11}^{12}\left(\mathbf{R_4}\right) + e^{i \left( k_x \frac{a}{2} - k_y \frac{a\sqrt{3}}{2} \right)} E_{11}^{12}\left(\mathbf{R_2}\right) + e^{-i \left( k_x \frac{a}{2} - k_y \frac{a\sqrt{3}}{2} \right)} E_{11}^{12}\left(\mathbf{R_5}\right) \\
	    & + e^{-i \left( k_x \frac{a}{2} + k_y \frac{a\sqrt{3}}{2} \right)} E_{11}^{12}\left(\mathbf{R_3}\right) + e^{i \left( k_x \frac{a}{2} + k_y \frac{a\sqrt{3}}{2} \right)} E_{11}^{12}\left(\mathbf{R_6}\right)
\end{align*}

trong đó
\begin{align*}
	E^{12}(\mathbf{R_2})
	 & = E^{12}(\sigma'_\nu \mathbf{R_1}) = D^1(\sigma'_\nu) E^{12}(\mathbf{R_1}) \left[ D^2(\sigma'_\nu)\right]^\dagger                                                                                                                                                                                                 \\
	 & = \begin{pmatrix}
		     1
	     \end{pmatrix}
	\begin{pmatrix}
		E_{11}^{12}(\mathbf{R_1}) & E_{12}^{12}(\mathbf{R_1})
	\end{pmatrix}
	\begin{pmatrix}
		\frac{1}{2}         & -\frac{\sqrt{3}}{2} \\
		-\frac{\sqrt{3}}{2} & -\frac{1}{2}
	\end{pmatrix}                                                                                                                                                                                                                                                             \\
	 & = \begin{pmatrix}
		     \dfrac{ E_{11}^{12}(\mathbf{R_1}) - \sqrt{3} E_{12}^{12}(\mathbf{R_1})}{2} & \dfrac{ - E_{11}^{12}(\mathbf{R_1})\sqrt{3} -  E_{12}^{12}(\mathbf{R_1})}{2}
	     \end{pmatrix}
\end{align*}
$\Rightarrow E_{11}^{12}(\mathbf{R_2}) =  \dfrac{ E_{11}^{12}(\mathbf{R_1}) - \sqrt{3} E_{12}^{12}(\mathbf{R_1})}{2} $

Tương tự ta có cho:

\begin{align*}
	E_{11}^{12}(\mathbf{R_3}) & =  \dfrac{ -E_{11}^{12}(\mathbf{R_1}) + \sqrt{3} E_{12}^{12}(\mathbf{R_1})}{2} \quad ;  E_{11}^{12}(\mathbf{R_4}) = - E_{11}^{12}(\mathbf{R_1})                                                 \\
	E_{11}^{12}(\mathbf{R_5}) & =  \dfrac{ -E_{11}^{12}(\mathbf{R_1}) - \sqrt{3} E_{12}^{12}(\mathbf{R_1})}{2} \quad ; E_{11}^{12}(\mathbf{R_6})  =  \dfrac{ E_{11}^{12}(\mathbf{R_1}) + \sqrt{3} E_{12}^{12}(\mathbf{R_1})}{2}
\end{align*}

\begin{align*}
	h_1 & = e^{i2\alpha} E_{11}^{12}(\mathbf{R_1}) - e^{i2\alpha} E_{11}^{12}(\mathbf{R_1})                                                                                                                     \\
	    & + e^{i(\alpha - \beta)}\dfrac{ E_{11}^{12}(\mathbf{R_1}) - \sqrt{3} E_{12}^{12}(\mathbf{R_1})}{2} + e^{-i(\alpha + \beta)}\dfrac{ -E_{11}^{12}(\mathbf{R_1}) + \sqrt{3} E_{12}^{12}(\mathbf{R_1})}{2} \\
	    & + e^{i(-\alpha + \beta)} \dfrac{ -E_{11}^{12}(\mathbf{R_1}) - \sqrt{3} E_{12}^{12}(\mathbf{R_1})}{2} + e^{i(\alpha+\beta)}\dfrac{ E_{11}^{12}(\mathbf{R_1}) + \sqrt{3} E_{12}^{12}(\mathbf{R_1})}{2}  \\
	    & = 2isin2\alpha E_{11}^{12}(\mathbf{R_1}) + 2i \dfrac{E_{11}^{12}(\mathbf{R_1})}{2} sin(\alpha-\beta) - 2\dfrac{E_{12}^{12}(\mathbf{R_1}\sqrt{3})}{2} cos(\alpha - \beta)                              \\
	    & +  2i \dfrac{E_{11}^{12}(\mathbf{R_1})}{2} sin(\alpha+\beta) + 2\dfrac{E_{12}^{12}(\mathbf{R_1}\sqrt{3})}{2} cos(\alpha + \beta)                                                                      \\
	    & = -2\sqrt{3} t_2 sin\alpha \; sin\beta + 2i t_1(sin2\alpha + sin\alpha\;\cos\beta)
\end{align*}

Đặt
\begin{align*}
	t_0 = E_{11}^{11}(\mathbf{R_1}); \quad
	t_1 = E_{11}^{12}(\mathbf{R_1}); \quad
	t_2 = E_{12}^{12}(\mathbf{R_1}); \quad \\
	t_{11} = E_{11}^{22}(\mathbf{R_1}); \quad
	t_{12} = E_{12}^{22}(\mathbf{R_1}); \quad
	t_{21} = E_{21}^{22}(\mathbf{R_1}); \quad
	t_{22} = E_{22}^{22}(\mathbf{R_1}); \quad
\end{align*}

\noindent\textbf{ $\ast$ h22}

\begin{align*}
	h_{22} & = \sum_{R}^{} e^{i\mathbf{k\cdot R}} E^{22}_{22}(\textbf{R})                                                                                                                                \\
	       & =  e^{i\mathbf{k\cdot R_1}} E^{22}_{22}(\mathbf{R_1}) +  e^{i\mathbf{k\cdot R_2}} E^{22}_{22}(\mathbf{R_2}) +  e^{i\mathbf{k\cdot R_3}} E^{22}_{22}(\mathbf{R_3})                           \\
	       & +  e^{i\mathbf{k\cdot R_4}} E^{22}_{22}(\mathbf{R_4}) +  e^{i\mathbf{k\cdot R_5}} E^{22}_{22}(\mathbf{R_5}) +  e^{i\mathbf{k\cdot R_6}} E^{22}_{22}(\mathbf{R_6}) + E^{22}_{22}(\mathbf{0})
\end{align*}

\begin{align*}
	E^{22} (\mathbf{R_2})
	 & =  E^{22} (\sigma'_\nu \mathbf{R_1})                                             \\
	 & = D^2(\sigma'_\nu)  E^{22}(\mathbf{R_1}) \left[ D^2(\sigma'_\nu) \right]^\dagger
\end{align*}

\[
	\renewcommand{\arraystretch}{0.75}
	=
	\begin{pmatrix}
		\frac{1}{2}         & -\frac{\sqrt{3}}{2} \\
		-\frac{\sqrt{3}}{2} & -\frac{1}{2}
	\end{pmatrix}
	\begin{pmatrix}
		E_{11}^{22}(\mathbf{R_1}) & E_{12}^{22}(\mathbf{R_1}) \\
		E_{21}^{22}(\mathbf{R_1}) & E_{22}^{22}(\mathbf{R_1})
	\end{pmatrix}
	\begin{pmatrix}
		\frac{1}{2}         & -\frac{\sqrt{3}}{2} \\
		-\frac{\sqrt{3}}{2} & -\frac{1}{2}
	\end{pmatrix}
\]

\[
	\renewcommand{\arraystretch}{0.75}
	=
	\begin{pmatrix}
		\frac{t_{11} - t_{12}\sqrt{3} -t_{21}\sqrt{3} + 3t_{22}}{4}   & \frac{-t_{11}\sqrt{3} - t_{12} + 3t_{21} + \sqrt{3}t_{22}}{4} \\
		\frac{-t_{11}\sqrt{3} + 3t_{12} - t_{21} + \sqrt{3}t_{22}}{4} & \frac{3t_{11} + t_{12}\sqrt{3} + c\sqrt{3} + t_{22}}{4}
	\end{pmatrix}
\]

$\Rightarrow E_{22}^{22}(\mathbf{R_2}) = \dfrac{3t_{11} + t_{12}\sqrt{3} + c\sqrt{3} + t_{22}}{4}$

Tương tự ta có cho:
\begin{align*}
	E_{22}^{22}(\mathbf{R_3}) & = \dfrac{3t_{11} - t_{12}\sqrt{3} - c\sqrt{3} + t_{22}}{4} \\
	E_{22}^{22}(\mathbf{R_4}) & = t_{22}                                                   \\
	E_{22}^{22}(\mathbf{R_5}) & = \dfrac{3t_{11} + t_{12}\sqrt{3} + c\sqrt{3} + t_{22}}{4} \\
	E_{22}^{22}(\mathbf{R_6}) & = \dfrac{3t_{11} - t_{12}\sqrt{3} - c\sqrt{3} + t_{22}}{4}
\end{align*}

Ta được:

\begin{align*}
	h_{22} & = e^{ i 2 \alpha} t_{22} + e^{- i 2 \alpha} t_{22}                                                                                                                                                                           \\
	       & + e^{i(\alpha - \beta)} \left(\dfrac{3t_{11} + t_{12}\sqrt{3} + c\sqrt{3} + t_{22}}{4}\right) +  e^{-i(\alpha + \beta)} \left(\dfrac{3t_{11} - t_{12}\sqrt{3} - c\sqrt{3} + t_{22}}{4}\right)                                \\
	       & + e^{i(-\alpha + \beta)} \left(\dfrac{3t_{11} + t_{12}\sqrt{3} + c\sqrt{3} + t_{22}}{4}\right) +  e^{i(\alpha + \beta)} \left(\dfrac{3t_{11} - t_{12}\sqrt{3} - c\sqrt{3} + t_{22}}{4}\right)                                \\
	       & = 2 cos(2\alpha) t_{22} + \dfrac{1}{4}3t_{11} \left(e^{i\alpha} + e^{-i\alpha}\right)\left(e^{-i\beta} + e^{i\beta}\right) + \dfrac{1}{4}t_{22} \left(e^{i\alpha} + e^{-i\alpha}\right)\left(e^{-i\beta} + e^{i\beta}\right) \\
	       & + c\sqrt{3}(e^{i(\alpha-\beta)} - e^{i(-\alpha+\beta)} +e^{i(-\alpha+\beta)} - e^{i(\alpha+\beta)})                                                                                                                          \\
	       & + t_{12}\sqrt{3}(e^{i(\alpha-\beta)} - e^{i(-\alpha+\beta)} + e^{i(-\alpha+\beta)} - e^{i(\alpha+\beta)})                                                                                                                    \\
	       & = 2cos(2\alpha) t_{22} + (3t_{11} + t_{22})cos\alpha \, cos\beta
\end{align*}
Sử dụng tính Hermite của Hamiltonian $h_{22}$ là số thực, nên $t_{12}= - t_{21}$

\noindent\textbf{*h11}
\begin{align*}
	H^{22}_{11} & = \sum_{\mathbf{R}} e^{i \mathbf{k} \cdot \mathbf{R}} E^{22}_{11}(\mathbf{R})
	= e^{i \mathbf{k} \cdot \mathbf{R_1}} E^{22}_{11}(\mathbf{R_1})
	+ e^{i \mathbf{k} \cdot \mathbf{R_2}} E^{22}_{11}(\mathbf{R_2})
	+ e^{i \mathbf{k} \cdot \mathbf{R_3}} E^{22}_{11}(\mathbf{R_3})                                                                                                                                                                 \\
	            & + e^{i \mathbf{k} \cdot \mathbf{R_4}} E^{22}_{11}(\mathbf{R_4})
	+ e^{i \mathbf{k} \cdot \mathbf{R_5}} E^{22}_{11}(\mathbf{R_5})
	+ e^{i \mathbf{k} \cdot \mathbf{R_6}} E^{22}_{11}(\mathbf{R_6})
	+ E^{22}_{11}(\mathbf{0})                                                                                                                                                                                                       \\
	            & = e^{i k_x a} E^{22}_{11}(\mathbf{R_1})
	+ e^{i \left( k_x \frac{a}{2} - k_y \frac{a\sqrt{3}}{2} \right)} E^{22}_{11}(\mathbf{R_2})
	+ e^{i \left( -k_x \frac{a}{2} - k_y \frac{a\sqrt{3}}{2} \right)} E^{22}_{11}(\mathbf{R_3})                                                                                                                                     \\
	            & + e^{-i k_x a } E^{22}_{11}(\mathbf{R_4})
	+ e^{i \left( -k_x \frac{a}{2} + k_y \frac{a\sqrt{3}}{2} \right)} E^{22}_{11}(\mathbf{R_5})
	+ e^{i \left( k_x \frac{a}{2} + k_y \frac{a\sqrt{3}}{2} \right)} E^{22}_{11}(\mathbf{R_6}) + \epsilon_2                                                                                                                         \\
	            & = e^{2i\alpha} t_{11} + e^{i \left( \alpha - \beta \right)} \frac{t_{11} -\sqrt{3}t_{12} - \sqrt{3}t_{21} + 3t_{22}}{4}                                                                                           \\
	            & + e^{i \left( -\alpha - \beta \right)} \frac{t_{11} -\sqrt{3}t_{12} - \sqrt{3}t_{21} + 3t_{22}}{4} + e^{-2i\alpha} t_{11}                                                                                         \\
	            & + e^{i \left( -\alpha + \beta \right)} \frac{t_{11} +\sqrt{3}t_{12} + \sqrt{3}t_{21} + 3t_{22}}{4} + e^{i \left( \alpha + \beta \right)} \frac{t_{11} +\sqrt{3}t_{12} + \sqrt{3}t_{21} + 3t_{22}}{4} + \epsilon_2 \\
	            & = 2t_{11}cos(2\alpha) + \left (t_{11} + 3t_{22}\right )cos(\alpha)cos(\beta) + \epsilon_2                                                                                                                         \\
	            & \textbf{Lưu ý ở đây đã sử dụng tính chất Hermite của $h_{11}$ phải là số thực}                                                                                                                                    \\ &\Rightarrow t_{12} = - t_{21}
\end{align*}

\begin{align*}
	E^{22} (\mathbf{R_2}) & = E^{22}(\sigma_\nu^{'} \mathbf{R_1}) = D^2(\sigma^{'}_\nu) E^{22} (\mathbf{R_1}) [D^2(\sigma_\nu ^{'})]^\dagger \\
	                      & =
	\begin{pmatrix}
		\frac{1}{2}         & -\frac{\sqrt{3}}{2} \\
		-\frac{\sqrt{3}}{2} & -\frac{1}{2}
	\end{pmatrix} \begin{pmatrix}
		              a & b \\
		              c & d
	              \end{pmatrix}\begin{pmatrix}
		                           \frac{1}{2}         & -\frac{\sqrt{3}}{2} \\
		                           -\frac{\sqrt{3}}{2} & -\frac{1}{2}
	                           \end{pmatrix}\text{Trong đó}\begin{pmatrix}
		                                                       a = t_{11} \\
		                                                       b = t_{12} \\
		                                                       c = t_{21} \\
		                                                       d = t_{22}
	                                                       \end{pmatrix}                                                      \\
	                      & \Rightarrow E^{22}_{11} (\mathbf{R_2}) = \frac{a - \sqrt{3}b - \sqrt{3}c + 3d}{4}
\end{align*}
Tương tự ta tìm được:
\begin{align*}
	E^{22}_{11} (\mathbf{R_3}) & = \frac{a - \sqrt{3}b - \sqrt{3}c + 3d}{4} \\
	E^{22}_{11} (\mathbf{R_4}) & = a                                        \\
	E^{22}_{11} (\mathbf{R_5}) & = \frac{a + \sqrt{3}b + \sqrt{3}c + 3d}{4} \\
	E^{22}_{11} (\mathbf{R_6}) & = \frac{a + \sqrt{3}b + \sqrt{3}c + 3d}{4} \\
\end{align*}

\noindent\textbf{*h12}
\begin{align*}
	H^{22}_{12} & = \sum_{\mathbf{R}} e^{i \mathbf{k} \cdot \mathbf{R}} E^{22}_{12}(\mathbf{R})                             \\
	            & = e^{i \mathbf{k} \cdot \mathbf{R_1}} E^{22}_{12}(\mathbf{R_1})
	+ e^{i \mathbf{k} \cdot \mathbf{R_2}} E^{22}_{12}(\mathbf{R_2})
	+ e^{i \mathbf{k} \cdot \mathbf{R_3}} E^{22}_{12}(\mathbf{R_3})                                                         \\
	            & \quad + e^{i \mathbf{k} \cdot \mathbf{R_4}} E^{22}_{12}(\mathbf{R_4})
	+ e^{i \mathbf{k} \cdot \mathbf{R_5}} E^{22}_{12}(\mathbf{R_5})
	+ e^{i \mathbf{k} \cdot \mathbf{R_6}} E^{22}_{12}(\mathbf{R_6})
	+ E^{22}_{12}(\mathbf{0})                                                                                               \\
	            & = e^{i k_x a} E^{22}_{12}(\mathbf{R_1})
	+ e^{i \left( k_x \frac{a}{2} - k_y \frac{a\sqrt{3}}{2} \right)} E^{22}_{12}(\mathbf{R_2})                              \\
	            & \quad + e^{i \left( -k_x \frac{a}{2} - k_y \frac{a\sqrt{3}}{2} \right)} E^{22}_{12}(\mathbf{R_3})         \\
	            & \quad + e^{-i k_x a } E^{22}_{12}(\mathbf{R_4})
	+ e^{i \left( -k_x \frac{a}{2} + k_y \frac{a\sqrt{3}}{2} \right)} E^{22}_{12}(\mathbf{R_5})                             \\
	            & \quad + e^{i \left( k_x \frac{a}{2} + k_y \frac{a\sqrt{3}}{2} \right)} E^{22}_{12}(\mathbf{R_6})          \\
	            & = e^{2i\alpha} t_{12}
	+ e^{i \left( \alpha - \beta \right)} \frac{-\sqrt{3}t_{11} - t_{12} + 3t_{21} + \sqrt{3}t_{22}}{4}                     \\
	            & \quad + e^{i \left( -\alpha - \beta \right)} \frac{\sqrt{3}t_{11} + t_{12} - 3t_{21} - \sqrt{3}t_{22}}{4} \\
	            & \quad - e^{-2i\alpha} t_{12}
	+ e^{i \left( -\alpha + \beta \right)} \frac{-\sqrt{3}t_{11} + t_{12} - 3t_{21} + \sqrt{3}t_{22}}{4}                    \\
	            & \quad + e^{i \left( \alpha + \beta \right)} \frac{\sqrt{3}t_{11} - t_{12} + 3t_{21} - \sqrt{3}t_{22}}{4}  \\
	            & = \sqrt{3}(t_{22} - t_{11}) \sin\alpha \sin\beta
	+ 4i t_{12} \sin\alpha \cos \alpha - i t_{12} \sin\alpha \cos\beta  + 3i t_{21} \sin\alpha \cos\beta
\end{align*}

\begin{align*}
	E^{22} (\mathbf{R_2}) & = E^{22}(\sigma_\nu^{'} \mathbf{R_1}) = D^2(\sigma^{'}_\nu) E^{22} (\mathbf{R_1}) [D^2(\sigma_\nu ^{'})]^\dagger \\
	                      & =
	\begin{pmatrix}
		\frac{1}{2}         & -\frac{\sqrt{3}}{2} \\
		-\frac{\sqrt{3}}{2} & -\frac{1}{2}
	\end{pmatrix} \begin{pmatrix}
		              a & b \\
		              c & d
	              \end{pmatrix}\begin{pmatrix}
		                           \frac{1}{2}         & -\frac{\sqrt{3}}{2} \\
		                           -\frac{\sqrt{3}}{2} & -\frac{1}{2}
	                           \end{pmatrix}\text{Trong đó}\begin{pmatrix}
		                                                       a = t_{11} \\
		                                                       b = t_{12} \\
		                                                       c = t_{21} \\
		                                                       d = t_{22}
	                                                       \end{pmatrix}                                                      \\
	                      & \Rightarrow E^{22}_{12} (\mathbf{R_2}) = \frac{-\sqrt{3}a - b + 3c + \sqrt{3}d}{4}
\end{align*}
Tương tự ta tìm được:
\begin{align*}
	E^{22}_{12} (\mathbf{R_3}) & = \frac{\sqrt{3}a + b - 3c - \sqrt{3}d}{4} \\
	E^{22}_{12} (\mathbf{R_4}) & = -b                                       \\
	E^{22}_{12} (\mathbf{R_5}) & = \frac{\sqrt{3}a + b - 3c + \sqrt{3}d}{4} \\
	E^{22}_{12} (\mathbf{R_6}) & = \frac{\sqrt{3}a - b + 3c - \sqrt{3}d}{4} \\
\end{align*}

\clearpage

Chọn hướng từ trường là $\renewcommand{\arraystretch}{0.7} B = \begin{pmatrix} 0  \\ 0  \\ B  \end{pmatrix}$. \\
\noindent Lại có $B = \vec{\nabla} \times \vec{A} =
	\begin{vmatrix}
		\vec{i}                      & \vec{j}                      & \vec{k}                      \\
		\dfrac{\partial}{\partial x} & \dfrac{\partial}{\partial y} & \dfrac{\partial}{\partial z} \\ A_x & A_y & A_z
	\end{vmatrix}\\
	= (\dfrac{\partial}{\partial y}A_z - \dfrac{\partial}{\partial z}A_y)\vec{i} + (\dfrac{\partial}{\partial z}A_x - \dfrac{\partial}{\partial x}A_z)\vec{j} + (\dfrac{\partial}{\partial x}A_y - \dfrac{\partial}{\partial y}A_x)\vec{k}$

\noindent Có thể chọn $A = \renewcommand{\arraystretch}{0.7} \begin{pmatrix} 0  \\ B\cdot x  \\ 0  \end{pmatrix}$

\begin{align*}
	H_{\mu \mu'}^{jj'}(\textbf{k}) = \sum_{\mu\mu' jj'} \sum_{\textbf{R}} e^{\frac{ie}{\hbar}\int_{0}^{\textbf{R}} \textbf{A}(\mathbf{r'})d\mathbf{r'} } e^{i \mathbf{k}\cdot\textbf{R}} E_{\mu\mu'}^{jj'}(\textbf{R})
\end{align*}

\noindent $\ast$ \textbf{h0}

\begin{align*}
	h_0 = H_{11}^{11}(\textbf{k}) & = \sum_{\textbf{R}} e^{\frac{ie}{\hbar}\int_{0}^{\mathbf{R}}A(\mathbf{r'})d\mathbf{r'}}e^{i\mathbf{k\cdot R}} E_{11}^{11}(\mathbf{R})                                                                                                               \\
	                              & = e^{\frac{ie}{\hbar}\int_{0}^{\mathbf{R_1}}A(\mathbf{r'})d\mathbf{r'}}e^{i\mathbf{k\cdot R_1}} E_{11}^{11}(\mathbf{R_1}) + e^{\frac{ie}{\hbar}\int_{0}^{\mathbf{R_2}}A(\mathbf{r'})d\mathbf{r'}}e^{i\mathbf{k\cdot R_2}} E_{11}^{11}(\mathbf{R_2}) \\
	                              & + e^{\frac{ie}{\hbar}\int_{0}^{\mathbf{R_3}}A(\mathbf{r'})d\mathbf{r'}}e^{i\mathbf{k\cdot R_3}} E_{11}^{11}(\mathbf{R_3}) + e^{\frac{ie}{\hbar}\int_{0}^{\mathbf{R_4}}A(\mathbf{r'})d\mathbf{r'}}e^{i\mathbf{k\cdot R_4}} E_{11}^{11}(\mathbf{R_4}) \\
	                              & + e^{\frac{ie}{\hbar}\int_{0}^{\mathbf{R_5}}A(\mathbf{r'})d\mathbf{r'}}e^{i\mathbf{k\cdot R_5}} E_{11}^{11}(\mathbf{R_5}) + e^{\frac{ie}{\hbar}\int_{0}^{\mathbf{R_6}}A(\mathbf{r'})d\mathbf{r'}}e^{i\mathbf{k\cdot R_6}} E_{11}^{11}(\mathbf{R_6})
\end{align*}

\noindent Xét $e^{\frac{ie}{\hbar}\int_{0}^{\mathbf{R}}A(\mathbf{r'})d\mathbf{r'}}$

\noindent Đặt $A = \left(P(x,y),Q(x,y),R(x,y)\right) = (0,Bx,0)$

\noindent Phương trình tham số cho $x,y$:

\begin{align*}
	  & x = x(t) = x_0 + \alpha t                                            \\
	  & y = y(t) = y_0 + \beta t                                             \\
	C & \; \text{là đường cong đi từ} \; \mathbf{R_0} \rightarrow \mathbf{R}
\end{align*}

\clearpage

\noindent $\ast \underset{(0,0)}{\mathbf{R_0}}  \longrightarrow \underset{(a,0)}{\mathbf{R_1}}$

Ta có:
\begin{align*}
	 & x = at                                                                                                                                                                  \\
	 & y = 0                                                                                                                                                                   \\
	 & \Rightarrow \int_{C} \mathbf{A}(\mathbf{r'})\cdot d\mathbf{r'} =  \int_{t_B}^{t_A} \left[ P(x,y)\dfrac{dx}{dt} + Q(x,y)\dfrac{dy}{dt} + R(x,y)\dfrac{dz}{dt} \right] dt \\
	 & = \int_{0}^{1} \left[ 0\dfrac{dx}{dt} + Bx0 + 0\dfrac{dz}{dt} \right] dt = 0
\end{align*}

\noindent $\ast \underset{(0,0)}{\mathbf{R_0}}  \longrightarrow \underset{(\frac{a}{2},-\frac{a\sqrt{3}}{2})}{\mathbf{R_2}}$

Ta có:
\begin{align*}
	 & x = \dfrac{a}{2}t                                                                                                                                                       \\
	 & y = -\dfrac{a\sqrt{3}}{2}t                                                                                                                                              \\
	 & \Rightarrow \int_{C} \mathbf{A}(\mathbf{r'})\cdot d\mathbf{r'} =  \int_{t_B}^{t_A} \left[ P(x,y)\dfrac{dx}{dt} + Q(x,y)\dfrac{dy}{dt} + R(x,y)\dfrac{dz}{dt} \right] dt \\
	 & = \int_{0}^{1} \left[ 0\dfrac{dx}{dt} + Bx\left(-\dfrac{a\sqrt{3}}{2}\right) + 0\dfrac{dz}{dt} \right] dt = -\dfrac{Ba^2\sqrt{3}}{8}
\end{align*}

\noindent $\ast \underset{(0,0)}{\mathbf{R_0}}  \longrightarrow \underset{(-\frac{a}{2},-\frac{a\sqrt{3}}{2})}{\mathbf{R_3}}$

Ta có:
\begin{align*}
	 & x = -\dfrac{a}{2}t                                                                                                                                                                           \\
	 & y = -\dfrac{a\sqrt{3}}{2}t                                                                                                                                                                   \\
	 & \Rightarrow \int_{C} \mathbf{A}(\mathbf{r'})\cdot d\mathbf{r'} =  \int_{t_B}^{t_A} \left[ P(x,y)\dfrac{dx}{dt} + Q(x,y)\dfrac{dy}{dt} + R(x,y)\dfrac{dz}{dt} \right] dt                      \\
	 & = \int_{0}^{1} \left[ 0\dfrac{dx}{dt} + Bx\left(-\dfrac{a\sqrt{3}}{2}\right) + 0\dfrac{dz}{dt} \right] dt = B \left( -\dfrac{a}{2}\right)\left(-\dfrac{a\sqrt{3}}{2}\right) \int_{0}^{1}t dt \\
	 & = \dfrac{Ba^2\sqrt{3}}{8}
\end{align*}

\clearpage

Xét $\mathbf{R_4},\mathbf{R_5},\mathbf{R_6}$: ta nhận thấy có thể đưa đường
cong $\mathbf{C}$ từ $\mathbf{R_0}$ cho tới $\mathbf{R}$ về các dạng của
$\mathbf{R_1},\mathbf{R_2},\mathbf{R_3}$. Lúc này đường cong sẽ là
$\mathbf{-C}$ \\ Dựa vào tính chất của tích phân đường:
\begin{align*}
	\int_{C}^{}\vec{f}d\vec{\mathbf{r}}              & = - \int_{-C}^{}\vec{f}d\vec{\mathbf{r}} \\
	\Rightarrow -\int_{C}^{}\vec{f}d\vec{\mathbf{r}} & = \int_{-C}^{}\vec{f}d\vec{\mathbf{r}}   \\
\end{align*}

\noindent $\ast \underset{(0,0)}{\mathbf{R_0}}  \longrightarrow \underset{(0,-a)}{\mathbf{R_4}}$

Ta có:
\begin{align*}
	 & x = - at                                                                                                                                                                                                                       \\
	 & y = 0                                                                                                                                                                                                                          \\
	 & \Rightarrow \int_{-C} \mathbf{A}(\mathbf{r'})\cdot d\mathbf{r'} =-\int_{C} \mathbf{A}(\mathbf{r'})\cdot d\mathbf{r'}=  - \int_{t_B}^{t_A} \left[ P(x,y)\dfrac{dx}{dt} + Q(x,y)\dfrac{dy}{dt} + R(x,y)\dfrac{dz}{dt} \right] dt \\
	 & = - \int_{0}^{1} \left[ 0\dfrac{dx}{dt} + Bx0 + 0\dfrac{dz}{dt} \right] dt = 0
\end{align*}

\noindent $\ast \underset{(0,0)}{\mathbf{R_0}}  \longrightarrow \underset{(-\frac{a}{2},\frac{a\sqrt{3}}{2})}{\mathbf{R_5}}$

Ta có:
\begin{align*}
	 & x = -\dfrac{a}{2}t                                                                                                                                                                                                            \\
	 & y = \dfrac{a\sqrt{3}}{2}t                                                                                                                                                                                                     \\
	 & \Rightarrow \int_{-C} \mathbf{A}(\mathbf{r'})\cdot d\mathbf{r'} =-\int_{C} \mathbf{A}(\mathbf{r'})\cdot d\mathbf{r'} =  \int_{t_B}^{t_A} \left[ P(x,y)\dfrac{dx}{dt} + Q(x,y)\dfrac{dy}{dt} + R(x,y)\dfrac{dz}{dt} \right] dt \\
	 & = -\int_{0}^{1} \left[ 0\dfrac{dx}{dt} + Bx\dfrac{a\sqrt{3}}{2} + 0\dfrac{dz}{dt} \right] dt = \dfrac{Ba^2\sqrt{3}}{8}
\end{align*}

\clearpage

\noindent $\ast \underset{(0,0)}{\mathbf{R_0}}  \longrightarrow \underset{(\frac{a}{2},\frac{a\sqrt{3}}{2})}{\mathbf{R_6}}$

Ta có:
\begin{align*}
	 & x = \frac{a}{2}t                                                                                                                                                                                                              \\
	 & y = \frac{a\sqrt{3}}{2}t                                                                                                                                                                                                      \\
	 & \Rightarrow \int_{-C} \mathbf{A}(\mathbf{r'})\cdot d\mathbf{r'} =-\int_{C} \mathbf{A}(\mathbf{r'})\cdot d\mathbf{r'} =  \int_{t_B}^{t_A} \left[ P(x,y)\dfrac{dx}{dt} + Q(x,y)\dfrac{dy}{dt} + R(x,y)\dfrac{dz}{dt} \right] dt \\
	 & = -\int_{0}^{1} \left[ 0\dfrac{dx}{dt} + Bx \frac{a\sqrt{3}}{2} + 0\dfrac{dz}{dt} \right] dt = -\dfrac{Ba^2\sqrt{3}}{8}
\end{align*}

\noindent Vậy $h_0$ có dạng:

\begin{align*}
	h_0 = H_{11}^{11}(\textbf{k}) & = e^{0}e^{i\mathbf{k\cdot R_1}} E_{11}^{11}(\mathbf{R_1}) + e^{-\frac{ie}{\hbar}\frac{Ba^2\sqrt{3}}{8}}e^{i\mathbf{k\cdot R_2}} E_{11}^{11}(\mathbf{R_2})                                                                                                         \\
	                              & + e^{\frac{ie}{\hbar}\frac{Ba^2\sqrt{3}}{8}}e^{i\mathbf{k\cdot R_3}} E_{11}^{11}(\mathbf{R_3}) + e^{0}e^{i\mathbf{k\cdot R_4}} E_{11}^{11}(\mathbf{R_4})                                                                                                          \\
	                              & + e^{\frac{ie}{\hbar}\frac{Ba^2\sqrt{3}}{8}}e^{i\mathbf{k\cdot R_5}} E_{11}^{11}(\mathbf{R_5}) + e^{-\frac{ie}{\hbar}\frac{Ba^2\sqrt{3}}{8}}e^{i\mathbf{k\cdot R_6}} E_{11}^{11}(\mathbf{R_6}) + \epsilon_1                                                       \\
	                              & = e^{ik_xa}E_{11}^{11}(\mathbf{R_1}) + e^{-ik_xa}E_{11}^{11}(\mathbf{R_4}) + e^{-\frac{ie}{\hbar}\frac{Ba^2\sqrt{3}}{8}}e^{i\left(k_x\frac{a}{2} - k_y\frac{a\sqrt{3}}{2}\right)} E_{11}^{11}(\mathbf{R_2})                                                       \\
	                              & + e^{\frac{ie}{\hbar}\frac{Ba^2\sqrt{3}}{8}}e^{i\left(-k_x\frac{a}{2} - k_y\frac{a\sqrt{3}}{2}\right)} E_{11}^{11}(\mathbf{R_3}) + e^{\frac{ie}{\hbar}\frac{Ba^2\sqrt{3}}{8}}e^{i\left(-k_x\frac{a}{2} + k_y\frac{a\sqrt{3}}{2}\right)} E_{11}^{11}(\mathbf{R_5}) \\
	                              & + e^{-\frac{ie}{\hbar}\frac{Ba^2\sqrt{3}}{8}}e^{i\left(k_x\frac{a}{2} + k_y\frac{a\sqrt{3}}{2}\right)} E_{11}^{11}(\mathbf{R_6}) + \epsilon_1
\end{align*}

\noindent Đặt $k_x\dfrac{a}{2} = \alpha,\quad k_y\dfrac{a\sqrt{3}}{2} = \beta, \quad \dfrac{e}{\hbar}\dfrac{Ba^2\sqrt{3}}{8} = \eta ,\quad \alpha - \beta = \delta,\quad \alpha + \beta = \gamma$ và áp dụng các toán tử quay để biểu diễn ${\mathbf{R}}_1$ theo $\mathbf{R_1}$.

\begin{align*}
	E^{11}(\mathbf{R_4}) & = E^{11}(\sigma''\mathbf{R_4}) = D^1(\sigma'') E^{11}(\mathbf{R_1}) \left[D^1(\sigma'')\right]^\dagger = E^{11}(\mathbf{R_1}) \\
	E^{11}(\mathbf{R_2}) & = E^{11}(\sigma'\mathbf{R_1}) = D^1(\sigma') E^{11}(\mathbf{R_1}) \left[D^1(\sigma')\right]^\dagger = E^{11}(\mathbf{R_1})    \\
	E^{11}(\mathbf{R_3}) & = E^{11}(C_3^2\mathbf{R_1}) = D^1(C_3^2) E^{11}(\mathbf{R_1}) \left[D^1(C_3^2)\right]^\dagger = E^{11}(\mathbf{R_1})          \\
	E^{11}(\mathbf{R_5}) & = E^{11}(C_3\mathbf{R_1}) = D^1(C_3) E^{11}(\mathbf{R_1}) \left[D^1(C_3)\right]^\dagger = E^{11}(\mathbf{R_1})                \\
	E^{11}(\mathbf{R_6}) & = E^{11}(\sigma\mathbf{R_1}) = D^1(\sigma) E^{11}(\mathbf{R_1}) \left[D^1(\sigma)\right]^\dagger = E^{11}(\mathbf{R_1})
\end{align*}

\begin{align*}
	\Rightarrow h_0 & = 2 E_{11}^{11}(\mathbf{R_1}) \cos (2\alpha) + (e^{-i\eta}e^{i\delta} + e^{i\eta}e^{-i\gamma} + e^{i\eta}e^{-i\delta} + e^{-i\eta}e^{i\gamma} ) E_{11}^{11}(\mathbf{R_1}) +\epsilon_1 \\
	                & = 2 E_{11}^{11}(\mathbf{R_1}) \cos (2\alpha) + E_{11}^{11}(\mathbf{R_1}) \left[(\cos\eta - i\sin\eta ) e^{i\delta} + (\cos\eta + i\sin\eta )e^{-i\delta} \right]                      \\
	                & + E_{11}^{11}\mathbf{R_1} \left[(\cos\eta + i \sin\eta)e^{-i\gamma} + (\cos\eta - i \sin\eta)e^{i\gamma}\right] +  \epsilon_1                                                         \\
	                & = 2 E_{11}^{11}(\mathbf{R_1}) \cos (2\alpha) + E_{11}^{11}(\mathbf{R_1}) \left[2\cos\eta \cos\delta - i\sin\eta( 2i\sin\delta)  \right]                                               \\
	                & + E_{11}^{11}(\mathbf{R_1}) \left[2\cos\eta \cos\gamma - i\sin\eta( 2i\sin\gamma)  \right] +   \epsilon_1                                                                             \\
	                & = 2 E_{11}^{11}(\mathbf{R_1}) \cos (2\alpha) + 2E_{11}^{11}(\mathbf{R_1}) \left[\cos\eta (\cos\gamma + \cos\delta) + \sin\eta(\sin\gamma + \sin\delta)\right] +   \epsilon_1          \\
	                & = 2 t_0 \left[ \cos (2\alpha) + 2\cos\eta \cos\alpha\cos\beta + 2\sin\eta\sin\alpha\cos\beta\right] +   \epsilon_1
\end{align*}

\noindent $\ast$ \textbf{h1}
\begin{align*}
	h_1 = H_{11}^{12}(\textbf{k}) & = \sum_{\textbf{R}} e^{\frac{ie}{\hbar}\int_{0}^{\mathbf{R}}A(\mathbf{r'})d\mathbf{r'}}e^{i\mathbf{k\cdot R}} E_{11}^{12}(\mathbf{R})                                                                                                               \\
	                              & = e^{\frac{ie}{\hbar}\int_{0}^{\mathbf{R_1}}A(\mathbf{r'})d\mathbf{r'}}e^{i\mathbf{k\cdot R_1}} E_{11}^{12}(\mathbf{R_1}) + e^{\frac{ie}{\hbar}\int_{0}^{\mathbf{R_2}}A(\mathbf{r'})d\mathbf{r'}}e^{i\mathbf{k\cdot R_2}} E_{11}^{12}(\mathbf{R_2}) \\
	                              & + e^{\frac{ie}{\hbar}\int_{0}^{\mathbf{R_3}}A(\mathbf{r'})d\mathbf{r'}}e^{i\mathbf{k\cdot R_3}} E_{11}^{12}(\mathbf{R_3}) + e^{\frac{ie}{\hbar}\int_{0}^{\mathbf{R_4}}A(\mathbf{r'})d\mathbf{r'}}e^{i\mathbf{k\cdot R_4}} E_{11}^{12}(\mathbf{R_4}) \\
	                              & + e^{\frac{ie}{\hbar}\int_{0}^{\mathbf{R_5}}A(\mathbf{r'})d\mathbf{r'}}e^{i\mathbf{k\cdot R_5}} E_{11}^{12}(\mathbf{R_5}) + e^{\frac{ie}{\hbar}\int_{0}^{\mathbf{R_6}}A(\mathbf{r'})d\mathbf{r'}}e^{i\mathbf{k\cdot R_6}} E_{11}^{12}(\mathbf{R_6})
\end{align*}

Trong đó:
\begin{align*}
	\ast E^{12}(\mathbf{R_4})             & = E^{12}(\sigma''\mathbf{R_4}) = D^1(\sigma'') E^{12}(\mathbf{R_1}) \left[D^2(\sigma'')\right]^\dagger \\
	                                      & = 1
	\begin{pmatrix}
		E_{11}^{12}(\mathbf{R_1}) & E_{12}^{12}(\mathbf{R_1})
	\end{pmatrix}
	\begin{pmatrix}
		-1 & 0 \\
		0  & 1
	\end{pmatrix}                                                                                                                                 \\
	                                      & =
	\begin{pmatrix}
		- E_{11}^{12}(\mathbf{R_1}) & E_{12}^{12}(\mathbf{R_1})
	\end{pmatrix}                                                                                        \\
	\Rightarrow E_{11}^{12}(\mathbf{R_4}) & = - E_{11}^{12}(\mathbf{R_1}) , \quad E_{12}^{12}(\mathbf{R_4}) = E_{11}^{12}(\mathbf{R_1})
\end{align*}

\begin{align*}
	\ast                                  & E^{12}(\mathbf{R_2})             = E^{12}(\sigma'\mathbf{R_2}) = D^1(\sigma') E^{12}(\mathbf{R_1}) \left[D^2(\sigma')\right]^\dagger \\
	                                      & = 1
	\begin{pmatrix}
		E_{11}^{12}(\mathbf{R_1}) & E_{12}^{12}(\mathbf{R_1})
	\end{pmatrix}
	\begin{pmatrix}
		\frac{1}{2}         & -\frac{\sqrt{3}}{2} \\
		-\frac{\sqrt{3}}{2} & -\frac{1}{2}
	\end{pmatrix}                                                                                                                     \\
	                                      & =
	\begin{pmatrix}
		\dfrac{E_{11}^{12}(\mathbf{R_1}) - \sqrt{3}E_{12}^{12}(\mathbf{R_1})}{2} & \dfrac{ -\sqrt{3}E_{11}^{12}(\mathbf{R_1}) - E_{12}^{12}(\mathbf{R_1})}{2}
	\end{pmatrix}                        \\
	\Rightarrow E_{11}^{12}(\mathbf{R_2}) & = \dfrac{E_{11}^{12}(\mathbf{R_1}) - \sqrt{3}E_{12}^{12}(\mathbf{R_1})}{2}                                                           \\
	\quad E_{12}^{12}(\mathbf{R_2})       & = \dfrac{ -\sqrt{3}E_{11}^{12}(\mathbf{R_1}) - E_{12}^{12}(\mathbf{R_1})}{2}
\end{align*}

Một cách tương tự ta có cho:
\begin{align*}
	E_{11}^{12}(\mathbf{R_3}) & =  \dfrac{ -E_{11}^{12}(\mathbf{R_1}) + \sqrt{3} E_{12}^{12}(\mathbf{R_1})}{2} \quad ;  E_{11}^{12}(\mathbf{R_4}) = - E_{11}^{12}(\mathbf{R_1})                                                 \\
	E_{11}^{12}(\mathbf{R_5}) & =  \dfrac{ -E_{11}^{12}(\mathbf{R_1}) - \sqrt{3} E_{12}^{12}(\mathbf{R_1})}{2} \quad ; E_{11}^{12}(\mathbf{R_6})  =  \dfrac{ E_{11}^{12}(\mathbf{R_1}) + \sqrt{3} E_{12}^{12}(\mathbf{R_1})}{2}
\end{align*}

\begin{align*}
	h_1  =           & E_{11}^{12}(\mathbf{R_1})\left( e^{ik_x a} - e^{-ik_x a} \right) + e^{-i\eta} e^{i\delta} \dfrac{E_{11}^{12}(\mathbf{R_1}) - \sqrt{3}E_{12}^{12}(\mathbf{R_1})}{2}                                                      \\
	                 & + e^{i\eta} e^{-i\gamma} \dfrac{ -E_{11}^{12}(\mathbf{R_1}) + \sqrt{3} E_{12}^{12}(\mathbf{R_1})}{2} + e^{i\eta} e^{-i\delta} \dfrac{ -E_{11}^{12}(\mathbf{R_1}) - \sqrt{3} E_{12}^{12}(\mathbf{R_1})}{2}               \\
	                 & + e^{-i\eta} e^{i\gamma} \dfrac{ E_{11}^{12}(\mathbf{R_1}) + \sqrt{3} E_{12}^{12}(\mathbf{R_1})}{2}                                                                                                                     \\
	=                & E_{11}^{12}(\mathbf{R_1})\left( e^{ik_x a} - e^{-ik_x a} \right) + \dfrac{E_{11}^{12}(\mathbf{R_1})}{2}\left( e^{-i\eta} e^{i\delta} - e^{i\eta} e^{-i\gamma} - e^{i\eta} e^{-i\delta} + e^{-i\eta} e^{i\gamma} \right) \\
	                 & + \dfrac{\sqrt{3}E_{12}^{12}(\mathbf{R_1})}{2}\left( -e^{-i\eta} e^{i\delta} + e^{i\eta} e^{-i\gamma} - e^{i\eta} e^{-i\delta} + e^{-i\eta} e^{i\gamma} \right)                                                         \\
	=                & E_{11}^{12}(\mathbf{R_1})\left( 2i\sin2\alpha \right) + \dfrac{E_{11}^{12}(\mathbf{R_1})}{2}4i(\cos\eta\sin\alpha\cos\beta - \sin\eta\cos\alpha\cos\beta)                                                               \\
	                 & +\dfrac{\sqrt{3}E_{12}^{12}(\mathbf{R_1})}{2}4(-\cos\eta\sin\alpha\sin\beta + \sin\eta\sin\alpha\cos\beta)                                                                                                              \\
	\Rightarrow h_1= & 2 i t_1 (\sin 2\alpha + \cos\eta\sin\alpha\cos\beta -    \sin\eta\cos\alpha\cos\beta )                                                                                                                                  \\
	                 & - 2\sqrt{3}t_2\left[ \cos\eta\sin\alpha\sin\beta - \sin\eta\cos\alpha\sin\beta  \right]
\end{align*}

\noindent$\ast$ \textbf{h2}
\begin{align*}
	h_2 = H_{12}^{12}(\textbf{k}) & = \sum_{\textbf{R}} e^{\frac{ie}{\hbar}\int_{0}^{\mathbf{R}}A(\mathbf{r'})d\mathbf{r'}}e^{i\mathbf{k\cdot R}} E_{12}^{12}(\mathbf{R})                                                                                                               \\
	                              & = e^{\frac{ie}{\hbar}\int_{0}^{\mathbf{R_1}}A(\mathbf{r'})d\mathbf{r'}}e^{i\mathbf{k\cdot R_1}} E_{12}^{12}(\mathbf{R_1}) + e^{\frac{ie}{\hbar}\int_{0}^{\mathbf{R_2}}A(\mathbf{r'})d\mathbf{r'}}e^{i\mathbf{k\cdot R_2}} E_{12}^{12}(\mathbf{R_2}) \\
	                              & + e^{\frac{ie}{\hbar}\int_{0}^{\mathbf{R_3}}A(\mathbf{r'})d\mathbf{r'}}e^{i\mathbf{k\cdot R_3}} E_{12}^{12}(\mathbf{R_3}) + e^{\frac{ie}{\hbar}\int_{0}^{\mathbf{R_4}}A(\mathbf{r'})d\mathbf{r'}}e^{i\mathbf{k\cdot R_4}} E_{12}^{12}(\mathbf{R_4}) \\
	                              & + e^{\frac{ie}{\hbar}\int_{0}^{\mathbf{R_5}}A(\mathbf{r'})d\mathbf{r'}}e^{i\mathbf{k\cdot R_5}} E_{12}^{12}(\mathbf{R_5}) + e^{\frac{ie}{\hbar}\int_{0}^{\mathbf{R_6}}A(\mathbf{r'})d\mathbf{r'}}e^{i\mathbf{k\cdot R_6}} E_{12}^{12}(\mathbf{R_6})
\end{align*}

\noindent Trong đó:
\begin{align*}
	E_{12}^{12}(\mathbf{R_2}) & = \dfrac{ - \sqrt{3}E_{11}^{12}(\mathbf{R_1}) - E_{12}^{12}(\mathbf{R_1})}{2}                                                                                                            \\
	E_{12}^{12}(\mathbf{R_3}) & = \dfrac{ - \sqrt{3}E_{11}^{12}(\mathbf{R_1}) - E_{12}^{12}(\mathbf{R_1})}{2} \quad ;  E_{12}^{12}(\mathbf{R_4}) = E_{11}^{12}(\mathbf{R_1})                                             \\
	E_{12}^{12}(\mathbf{R_5}) & = \dfrac{\sqrt{3}E_{11}^{12}(\mathbf{R_1}) - E_{12}^{12}(\mathbf{R_1})}{2} \quad ; E_{12}^{12}(\mathbf{R_6})  = \dfrac{\sqrt{3}E_{11}^{12}(\mathbf{R_1}) - E_{12}^{12}(\mathbf{R_1})}{2}
\end{align*}
Thế vô:
\begin{align*}
	h_2 = & E_{12}^{12}(\mathbf{R_1}) \left( e^{ik_x a} + e^{-ik_x a} \right) + e^{-i\eta} e^{i\delta} \dfrac{ - \sqrt{3}E_{11}^{12}(\mathbf{R_1}) - E_{12}^{12}(\mathbf{R_1})}{2}                                 \\
	      & + e^{i\eta} e^{-i\gamma} \dfrac{ - \sqrt{3}E_{11}^{12}(\mathbf{R_1}) - E_{12}^{12}(\mathbf{R_1})}{2} + e^{i\eta} e^{-i\delta} \dfrac{\sqrt{3}E_{11}^{12}(\mathbf{R_1}) - E_{12}^{12}(\mathbf{R_1})}{2} \\
	      & + e^{-i\eta} e^{i\gamma} \dfrac{\sqrt{3}E_{11}^{12}(\mathbf{R_1}) - E_{12}^{12}(\mathbf{R_1})}{2}                                                                                                      \\
	=     & 2 E_{12}^{12}(\mathbf{R_1}) \cos 2\alpha + \dfrac{\sqrt{3}E_{11}^{12}(\mathbf{R_1})}{2}\left( - e^{-i\eta}e^{i\delta} - e^{i\eta}e^{-i\gamma} + e^{i\eta}e^{-i\delta} + e^{-i\eta}e^{i\gamma} \right)  \\
	      & + \dfrac{E_{12}^{12}(\mathbf{R_1})}{2}\left( - e^{-i\eta}e^{i\delta} - e^{i\eta}e^{-i\gamma} - e^{i\eta}e^{-i\delta} - e^{-i\eta}e^{i\gamma} \right)                                                   \\
	=     & 2t_2 \cos 2\alpha + 2i\sqrt{3}t_1(\cos\eta\cos\alpha\sin\beta +  \sin\eta\sin\alpha\cos\beta)                                                                                                          \\
	      & - 2t_2 (\cos\eta\cos\alpha\cos\beta + \sin\eta\sin\alpha\sin\beta)                                                                                                                                     \\
	h_2 = & 2t_2 (\cos 2\alpha - \cos\eta\cos\alpha\cos\beta - \sin\eta\sin\alpha\cos\beta)     \nonumber                                                                                                          \\
	      & + 2i\sqrt{3}t_1(\cos\eta\cos\alpha\sin\beta +  \sin\eta\sin\alpha\sin\beta)
\end{align*}

Các ma trận $E^{22}(\mathbf{R})$
\begin{align*}
	\ast E^{22} (\mathbf{R_2})
	 & =  E^{22} (\sigma'_\nu \mathbf{R_1})                                             \\
	 & = D^2(\sigma'_\nu)  E^{22}(\mathbf{R_1}) \left[ D^2(\sigma'_\nu) \right]^\dagger \\
	 &
	=
	\begin{pmatrix}
		\renewcommand{\arraystretch}{0.72}
		\frac{1}{2}         & -\frac{\sqrt{3}}{2} \\
		-\frac{\sqrt{3}}{2} & -\frac{1}{2}
	\end{pmatrix}
	\begin{pmatrix}
		E_{11}^{22}(\mathbf{R_1}) & E_{12}^{22}(\mathbf{R_1}) \\
		E_{21}^{22}(\mathbf{R_1}) & E_{22}^{22}(\mathbf{R_1})
	\end{pmatrix}
	\begin{pmatrix}
		\frac{1}{2}         & -\frac{\sqrt{3}}{2} \\
		-\frac{\sqrt{3}}{2} & -\frac{1}{2}
	\end{pmatrix}                            \\
	 & =
	\begin{pmatrix}
		\frac{t_{11} - \sqrt{3}t_{12} - \sqrt{3}t_{21} + 3t_{22}}{4}  & \frac{-\sqrt{3}t_{11} - t_{12} + 3t_{21} + \sqrt{3}t_{22}}{4} \\
		\frac{-\sqrt{3}t_{11} + 3t_{12} - t_{21} + \sqrt{3}t_{22}}{4} & \frac{3t_{11} + \sqrt{3}t_{12} + \sqrt{3}t_{21} + t_{22}}{4}
	\end{pmatrix}
\end{align*}

\begin{align*}
	\Rightarrow E_{11}^{22}(\mathbf{R_2}) & = \frac{t_{11} - \sqrt{3}t_{12} - \sqrt{3}t_{21} + 3t_{22}}{4}  \\
	E_{12}^{22}(\mathbf{R_2})             & = \frac{-\sqrt{3}t_{11} - t_{12} + 3t_{21} + \sqrt{3}t_{22}}{4} \\
	E_{21}^{22}(\mathbf{R_2})             & = \frac{-\sqrt{3}t_{11} + 3t_{12} - t_{21} + \sqrt{3}t_{22}}{4} \\
	E_{22}^{22}(\mathbf{R_2})             & = \frac{3t_{11} + \sqrt{3}t_{12} + \sqrt{3}t_{21} + t_{22}}{4}  \\
\end{align*}

\begin{align*}
	\ast E^{22}(\mathbf{R_3}) & = E^{22}(C_3^2\mathbf{ R_1})                                                                                                   \\
	                          & = D^2(C_3^2)  E^{22}(\mathbf{R_1}) \left[ D^2(C_3^2) \right]^\dagger                                                           \\
	                          &
	=
	\begin{pmatrix}
		\renewcommand{\arraystretch}{0.72}
		-\frac{1}{2}        & \frac{\sqrt{3}}{2} \\
		-\frac{\sqrt{3}}{2} & -\frac{1}{2}
	\end{pmatrix}
	\begin{pmatrix}
		E_{11}^{22}(\mathbf{R_1}) & E_{12}^{22}(\mathbf{R_1}) \\
		E_{21}^{22}(\mathbf{R_1}) & E_{22}^{22}(\mathbf{R_1})
	\end{pmatrix}
	\begin{pmatrix}
		-\frac{1}{2}       & -\frac{\sqrt{3}}{2} \\
		\frac{\sqrt{3}}{2} & -\frac{1}{2}
	\end{pmatrix}                                                                                                   \\
	                          & = \begin{pmatrix}
		                              \frac{t_{11} - \sqrt{3}t_{12} - \sqrt{3}t_{21} + 3t_{22}}{4} & \frac{\sqrt{3}t_{11} + t_{12} - 3 t_{21} - \sqrt{3}t_{22}}{4} \\
		                              \frac{\sqrt{3}t_{11} - 3t_{12} + t_{21} - \sqrt{3}t_{22}}{4} & \frac{3t_{11} + \sqrt{3}t_{12} + \sqrt{3}t_{21} + t_{22}}{4}
	                              \end{pmatrix}
\end{align*}

\begin{align*}
	\Rightarrow E_{11}^{22}(\mathbf{R_3}) & = \frac{t_{11} - \sqrt{3}t_{12} - \sqrt{3}t_{21} + 3t_{22}}{4} \\
	E_{12}^{22}(\mathbf{R_3})             & = \frac{\sqrt{3}t_{11} + t_{12} - 3 t_{21} -\sqrt{3}t_{22}}{4} \\
	E_{21}^{22}(\mathbf{R_3})             & = \frac{\sqrt{3}t_{11} - 3t_{12} + t_{21} - \sqrt{3}t_{22}}{4} \\
	E_{22}^{22}(\mathbf{R_3})             & = \frac{3t_{11} + \sqrt{3}t_{12} + \sqrt{3}t_{21} + t_{22}}{4} \\
\end{align*}

\begin{align*}
	\ast E^{22}(\mathbf{R_5}) & = E^{22}(C_3\mathbf{ R_1})                                                                                                       \\
	                          & = D^2(C_3)  E^{22}(\mathbf{R_1}) \left[ D^2(C_3) \right]^\dagger                                                                 \\
	                          & =
	\begin{pmatrix}
		\renewcommand{\arraystretch}{0.72}
		-\frac{1}{2}       & -\frac{\sqrt{3}}{2} \\
		\frac{\sqrt{3}}{2} & -\frac{1}{2}
	\end{pmatrix}
	\begin{pmatrix}
		E_{11}^{22}(\mathbf{R_1}) & E_{12}^{22}(\mathbf{R_1}) \\
		E_{21}^{22}(\mathbf{R_1}) & E_{22}^{22}(\mathbf{R_1})
	\end{pmatrix}
	\begin{pmatrix}
		-\frac{1}{2}        & \frac{\sqrt{3}}{2} \\
		-\frac{\sqrt{3}}{2} & -\frac{1}{2}
	\end{pmatrix}                                                                                                     \\
	                          & = \begin{pmatrix}
		                              \frac{t_{11} + \sqrt{3}t_{12} + \sqrt{3}t_{21} + 3t_{22}}{4}  & \frac{-\sqrt{3}t_{11} + t_{12} - 3 t_{21} + \sqrt{3}t_{22}}{4} \\
		                              \frac{-\sqrt{3}t_{11} - 3t_{12} + t_{21} + \sqrt{3}t_{22}}{4} & \frac{3t_{11} - \sqrt{3}t_{12} - \sqrt{3}t_{21} + t_{22}}{4}
	                              \end{pmatrix}
\end{align*}

\begin{align*}
	\Rightarrow E_{11}^{22}(\mathbf{R_5}) & =  \frac{t_{11} + \sqrt{3}t_{12} + \sqrt{3}t_{21} + 3t_{22}}{4}  \\
	E_{12}^{22}(\mathbf{R_5})             & = \frac{-\sqrt{3}t_{11} + t_{12} - 3 t_{21} + \sqrt{3}t_{22}}{4} \\
	E_{21}^{22}(\mathbf{R_5})             & = \frac{-\sqrt{3}t_{11} - 3t_{12} + t_{21} + \sqrt{3}t_{22}}{4}  \\
	E_{22}^{22}(\mathbf{R_5})             & = \frac{3t_{11} - \sqrt{3}t_{12} - \sqrt{3}t_{21} + t_{22}}{4}   \\
\end{align*}

\begin{align*}
	\ast E^{22}(\mathbf{R_4}) & = E^{22}(\sigma''_\nu\mathbf{ R_1})                                               \\
	                          & = D^2(\sigma''_\nu)  E^{22}(\mathbf{R_1}) \left[ D^2(\sigma''_nu) \right]^\dagger \\
	                          &
	=
	\begin{pmatrix}
		\renewcommand{\arraystretch}{0.72}
		-1 & 0 \\
		0  & 1
	\end{pmatrix}
	\begin{pmatrix}
		E_{11}^{22}(\mathbf{R_1}) & E_{12}^{22}(\mathbf{R_1}) \\
		E_{21}^{22}(\mathbf{R_1}) & E_{22}^{22}(\mathbf{R_1})
	\end{pmatrix}
	\begin{pmatrix}
		-1 & 0 \\
		0  & 1
	\end{pmatrix}                                                                                                \\
	                          & = \begin{pmatrix}
		                              t_{11}  & -t_{12} \\
		                              -t_{21} & t_{22}
	                              \end{pmatrix}
\end{align*}

\begin{align*}
	\Rightarrow E_{11}^{22}(\mathbf{R_4}) & = t_{11}  \\
	E_{12}^{22}(\mathbf{R_4})             & = -t_{12} \\
	E_{21}^{22}(\mathbf{R_4})             & = -t_{21} \\
	E_{22}^{22}(\mathbf{R_4})             & = t_{22}  \\
\end{align*}

\begin{align*}
	\ast E^{22}(\mathbf{R_6}) & = E^{22}(\sigma_\nu\mathbf{ R_1})                                                                                               \\
	                          & = D^2(\sigma_\nu)  E^{22}(\mathbf{R_1}) \left[ D^2(\sigma_\nu) \right]^\dagger                                                  \\
	                          &
	=
	\begin{pmatrix}
		\renewcommand{\arraystretch}{0.72}
		\frac{1}{2}        & \frac{\sqrt{3}}{2} \\
		\frac{\sqrt{3}}{2} & -\frac{1}{2}
	\end{pmatrix}
	\begin{pmatrix}
		E_{11}^{22}(\mathbf{R_1}) & E_{12}^{22}(\mathbf{R_1}) \\
		E_{21}^{22}(\mathbf{R_1}) & E_{22}^{22}(\mathbf{R_1})
	\end{pmatrix}
	\begin{pmatrix}
		\frac{1}{2}        & \frac{\sqrt{3}}{2} \\
		\frac{\sqrt{3}}{2} & -\frac{1}{2}
	\end{pmatrix}                                                                                                     \\
	                          & = \begin{pmatrix}
		                              \frac{t_{11} + \sqrt{3}t_{12} + \sqrt{3}t_{21} + 3t_{22}}{4} & \frac{-\sqrt{3}t_{11} - t_{12} + 3 t_{21} - \sqrt{3}t_{22}}{4} \\
		                              \frac{\sqrt{3}t_{11} + 3t_{12} - t_{21} - \sqrt{3}t_{22}}{4} & \frac{3t_{11} - \sqrt{3}t_{12} - \sqrt{3}t_{21} + t_{22}}{4}
	                              \end{pmatrix}
\end{align*}

\begin{align*}
	\Rightarrow E_{11}^{22}(\mathbf{R_6}) & =  \frac{t_{11} + \sqrt{3}t_{12} + \sqrt{3}t_{21} + 3t_{22}}{4} \\
	E_{12}^{22}(\mathbf{R_6})             & = \frac{\sqrt{3}t_{11} - t_{12} + 3 t_{21} - \sqrt{3}t_{22}}{4} \\
	E_{21}^{22}(\mathbf{R_6})             & = \frac{\sqrt{3}t_{11} + 3t_{12} - t_{21} - \sqrt{3}t_{22}}{4}  \\
	E_{22}^{22}(\mathbf{R_6})             & = \frac{3t_{11} - \sqrt{3}t_{12} - \sqrt{3}t_{21} + t_{22}}{4}  \\
\end{align*}

\noindent$\ast$ \textbf{h11}
\begin{align*}
	h_{11} = H_{11}^{22}(\textbf{k})  = & \sum_{\textbf{R}} e^{\frac{ie}{\hbar}\int_{0}^{\mathbf{R}}A(\mathbf{r'})d\mathbf{r'}}e^{i\mathbf{k\cdot R}} E_{11}^{22}(\mathbf{R})                                                                                                                 \\
	=                                   & e^{\frac{ie}{\hbar}\int_{0}^{\mathbf{R_1}}A(\mathbf{r'})d\mathbf{r'}}e^{i\mathbf{k\cdot R_1}} E_{11}^{22}(\mathbf{R_1}) + e^{\frac{ie}{\hbar}\int_{0}^{\mathbf{R_2}}A(\mathbf{r'})d\mathbf{r'}}e^{i\mathbf{k\cdot R_2}} E_{11}^{22}(\mathbf{R_2})   \\
	                                    & + e^{\frac{ie}{\hbar}\int_{0}^{\mathbf{R_3}}A(\mathbf{r'})d\mathbf{r'}}e^{i\mathbf{k\cdot R_3}} E_{11}^{22}(\mathbf{R_3}) + e^{\frac{ie}{\hbar}\int_{0}^{\mathbf{R_4}}A(\mathbf{r'})d\mathbf{r'}}e^{i\mathbf{k\cdot R_4}} E_{11}^{22}(\mathbf{R_4}) \\
	                                    & + e^{\frac{ie}{\hbar}\int_{0}^{\mathbf{R_5}}A(\mathbf{r'})d\mathbf{r'}}e^{i\mathbf{k\cdot R_5}} E_{11}^{22}(\mathbf{R_5}) + e^{\frac{ie}{\hbar}\int_{0}^{\mathbf{R_6}}A(\mathbf{r'})d\mathbf{r'}}e^{i\mathbf{k\cdot R_6}} E_{11}^{22}(\mathbf{R_6})
\end{align*}
Trong đó:
\begin{align*}
	E_{11}^{22}(\mathbf{R_2}) & = \dfrac{t_{11} - \sqrt{3}t_{12} - \sqrt{3}t_{21} + 3t_{22}}{4}  \\
	E_{11}^{22}(\mathbf{R_3}) & = \dfrac{t_{11} - \sqrt{3}t_{12} - \sqrt{3}t_{21} + 3t_{22}}{4}  \\
	E_{11}^{22}(\mathbf{R_4}) & = t_{11}                                                         \\
	E_{11}^{22}(\mathbf{R_5}) & =  \dfrac{t_{11} + \sqrt{3}t_{12} + \sqrt{3}t_{21} + 3t_{22}}{4} \\
	E_{11}^{22}(\mathbf{R_6}) & =  \dfrac{t_{11} + \sqrt{3}t_{12} + \sqrt{3}t_{21} + 3t_{22}}{4}
\end{align*}

Thế vô:

\begin{align*}
	h_{11} = & t_{11}\left( e^{ik_x a} + e^{-ik_x a} \right) + e^{-i\eta}e^{i\delta} \dfrac{t_{11} - \sqrt{3}t_{12} - \sqrt{3}t_{21} + 3t_{22}}{4}                                           \\
	         & + e^{i\eta} e^{-i\gamma} \dfrac{t_{11} - \sqrt{3}t_{12} - \sqrt{3}t_{21} + 3t_{22}}{4} + e^{i\eta} e^{-i\delta} \dfrac{t_{11} + \sqrt{3}t_{12} + \sqrt{3}t_{21} + 3t_{22}}{4} \\
	         & + e^{-i\eta} e^{i\gamma} \dfrac{t_{11} + \sqrt{3}t_{12} + \sqrt{3}t_{21} + 3t_{22}}{4}
\end{align*}

Do tính Hermite của Hamiltonian, ta có thể đưa $t_{12} = - t_{21}$, nên
$h_{11}$ đơn giản thành:

\begin{align*}
	h_{11} =            & e^{-i\eta} e^{i\delta} \dfrac{t_{11} + 3t_{22}}{4} + e^{i\eta} e^{-i\gamma} \dfrac{t_{11} + 3t_{22}}{4} + e^{i\eta} e^{-i\delta} \dfrac{t_{11} + 3t_{22}}{4} + e^{-i\eta} e^{i\gamma} \dfrac{t_{11} + 3t_{22}}{4} \\
	                    & + t_{11}\left( e^{ik_x a} + e^{-ik_x a} \right) + \epsilon_2                                                                                                                                                      \\
	=                   & ( e^{-i\eta} e^{i\delta} + e^{i\eta} e^{-i\gamma} + e^{i\eta} e^{-i\delta} + e^{-i\eta} e^{i\gamma} ) \dfrac{t_{11} + 3t_{22}}{4} + t_{11}\left( e^{ik_x a} + e^{-ik_x a} \right) + \epsilon_2                    \\
	=                   & \dfrac{t_{11} + 3t_{22}}{2} \left[ 2\cos\eta\cos\alpha\cos\beta + 2\sin\eta\sin\alpha\cos\beta \right] + 2t_{11}\cos2\alpha + \epsilon_2                                                                          \\
	\Rightarrow h_{11}= & (t_{11} + 3t_{22}) \left[ \cos\eta\cos\alpha\cos\beta + \sin\eta\sin\alpha\cos\beta \right] + 2t_{11}\cos2\alpha + \epsilon_2
\end{align*}

$\ast$ \textbf{h22}

\begin{align*}
	h_{22} = H_{22}^{22}(\textbf{k})  = & \sum_{\textbf{R}} e^{\frac{ie}{\hbar}\int_{0}^{\mathbf{R}}A(\mathbf{r'})d\mathbf{r'}}e^{i\mathbf{k\cdot R}} E_{22}^{22}(\mathbf{R}) + \epsilon_2                                                                                                                 \\
	=                                   & e^{\frac{ie}{\hbar}\int_{0}^{\mathbf{R_1}}A(\mathbf{r'})d\mathbf{r'}}e^{i\mathbf{k\cdot R_1}} E_{22}^{22}(\mathbf{R_1}) + e^{\frac{ie}{\hbar}\int_{0}^{\mathbf{R_2}}A(\mathbf{r'})d\mathbf{r'}}e^{i\mathbf{k\cdot R_2}} E_{22}^{22}(\mathbf{R_2})                \\
	                                    & + e^{\frac{ie}{\hbar}\int_{0}^{\mathbf{R_3}}A(\mathbf{r'})d\mathbf{r'}}e^{i\mathbf{k\cdot R_3}} E_{22}^{22}(\mathbf{R_3}) + e^{\frac{ie}{\hbar}\int_{0}^{\mathbf{R_4}}A(\mathbf{r'})d\mathbf{r'}}e^{i\mathbf{k\cdot R_4}} E_{22}^{22}(\mathbf{R_4})              \\
	                                    & + e^{\frac{ie}{\hbar}\int_{0}^{\mathbf{R_5}}A(\mathbf{r'})d\mathbf{r'}}e^{i\mathbf{k\cdot R_5}} E_{22}^{22}(\mathbf{R_5}) + e^{\frac{ie}{\hbar}\int_{0}^{\mathbf{R_6}}A(\mathbf{r'})d\mathbf{r'}}e^{i\mathbf{k\cdot R_6}} E_{22}^{22}(\mathbf{R_6}) + \epsilon_2
\end{align*}

Trong đó:

\begin{align*}
	E_{22}^{22}(\mathbf{R_2}) & = \dfrac{3t_{11} + \sqrt{3}t_{12} + \sqrt{3}t_{21} + t_{22}}{4} \\
	E_{22}^{22}(\mathbf{R_3}) & = \dfrac{3t_{11} + \sqrt{3}t_{12} + \sqrt{3}t_{21} + t_{22}}{4} \\
	E_{22}^{22}(\mathbf{R_4}) & = t_{22}                                                        \\
	E_{22}^{22}(\mathbf{R_5}) & = \dfrac{3t_{11} - \sqrt{3}t_{12} - \sqrt{3}t_{21} + t_{22}}{4} \\
	E_{22}^{22}(\mathbf{R_6}) & = \dfrac{3t_{11} - \sqrt{3}t_{12} - \sqrt{3}t_{21} + t_{22}}{4} \\
\end{align*}
\begin{align*}
	h_{22} =            & e^{-i\eta} e^{i\delta} \dfrac{3t_{11} + \sqrt{3}t_{12} + \sqrt{3}t_{21} + t_{22}}{4} + e^{i\eta} e^{-i\gamma} \dfrac{3t_{11} + \sqrt{3}t_{12} + \sqrt{3}t_{21} + t_{22}}{4}                                       \\
	                    & + e^{i\eta} e^{-i\delta} \dfrac{3t_{11} - \sqrt{3}t_{12} - \sqrt{3}t_{21} + t_{22}}{4}  + e^{-i\eta} e^{i\gamma} \dfrac{3t_{11} - \sqrt{3}t_{12} - \sqrt{3}t_{21} + t_{22}}{4}                                    \\
	                    & + t_{22}\left( e^{ik_x a} + e^{-ik_x a} \right) + \epsilon_2                                                                                                                                                      \\
	=                   & e^{-i\eta} e^{i\delta} \dfrac{3t_{11} + t_{22}}{4} + e^{i\eta} e^{-i\gamma} \dfrac{3t_{11} + t_{22}}{4} + e^{i\eta} e^{-i\delta} \dfrac{3t_{11} + t_{22}}{4} + e^{-i\eta} e^{i\gamma} \dfrac{3t_{11} + t_{22}}{4} \\
	                    & + t_{22}\left( e^{ik_x a} + e^{-ik_x a} \right) + \epsilon_2                                                                                                                                                      \\
	=                   & ( e^{-i\eta} e^{i\delta} + e^{i\eta} e^{-i\gamma} + e^{i\eta} e^{-i\delta} + e^{-i\eta} e^{i\gamma}) \dfrac{3t_{11} + t_{22}}{4} + t_{11}\left( e^{ik_x a} + e^{-ik_x a} \right) + \epsilon_2                     \\
	=                   & \dfrac{3t_{11} + t_{22}}{2} \left[ 2\cos\eta\cos\alpha\cos\beta + 2\sin\eta\sin\alpha\cos\beta \right] + 2t_{22}\cos2\alpha +\epsilon_2                                                                           \\
	\Rightarrow h_{22}= & (3t_{11} + t_{22}) \left[ \cos\eta\cos\alpha\cos\beta + \sin\eta\sin\alpha\cos\beta \right] + 2t_{22}\cos2\alpha + \epsilon_2
\end{align*}

$\ast$ \textbf{h12}

\begin{align*}
	h_{12} = H_{12}^{22}(\textbf{k})  = & \sum_{\textbf{R}} e^{\frac{ie}{\hbar}\int_{0}^{\mathbf{R}}A(\mathbf{r'})d\mathbf{r'}}e^{i\mathbf{k\cdot R}} E_{12}^{22}(\mathbf{R}) + \epsilon_2                                                                                                                 \\
	=                                   & e^{\frac{ie}{\hbar}\int_{0}^{\mathbf{R_1}}A(\mathbf{r'})d\mathbf{r'}}e^{i\mathbf{k\cdot R_1}} E_{12}^{22}(\mathbf{R_1}) + e^{\frac{ie}{\hbar}\int_{0}^{\mathbf{R_2}}A(\mathbf{r'})d\mathbf{r'}}e^{i\mathbf{k\cdot R_2}} E_{12}^{22}(\mathbf{R_2})                \\
	                                    & + e^{\frac{ie}{\hbar}\int_{0}^{\mathbf{R_3}}A(\mathbf{r'})d\mathbf{r'}}e^{i\mathbf{k\cdot R_3}} E_{12}^{22}(\mathbf{R_3}) + e^{\frac{ie}{\hbar}\int_{0}^{\mathbf{R_4}}A(\mathbf{r'})d\mathbf{r'}}e^{i\mathbf{k\cdot R_4}} E_{12}^{22}(\mathbf{R_4})              \\
	                                    & + e^{\frac{ie}{\hbar}\int_{0}^{\mathbf{R_5}}A(\mathbf{r'})d\mathbf{r'}}e^{i\mathbf{k\cdot R_5}} E_{12}^{22}(\mathbf{R_5}) + e^{\frac{ie}{\hbar}\int_{0}^{\mathbf{R_6}}A(\mathbf{r'})d\mathbf{r'}}e^{i\mathbf{k\cdot R_6}} E_{12}^{22}(\mathbf{R_6}) + \epsilon_2
\end{align*}

\noindent Trong đó:
\begin{align*}
	E_{12}^{22}(\mathbf{R_2}) & = \frac{-\sqrt{3}t_{11} - t_{12} + 3t_{21} + \sqrt{3}t_{22}}{4}  \\
	E_{12}^{22}(\mathbf{R_3}) & = \frac{\sqrt{3}t_{11} + t_{12} - 3 t_{21} -\sqrt{3}t_{22}}{4}   \\
	E_{12}^{22}(\mathbf{R_4}) & = -t_{12}                                                        \\
	E_{12}^{22}(\mathbf{R_5}) & = \frac{-\sqrt{3}t_{11} + t_{12} - 3 t_{21} + \sqrt{3}t_{22}}{4} \\
	E_{12}^{22}(\mathbf{R_6}) & = \frac{\sqrt{3}t_{11} - t_{12} + 3 t_{21} - \sqrt{3}t_{22}}{4}  \\
\end{align*}

\noindent Thế vô:
\begin{align*}
	h_{12} =             & e^{-i\eta} e^{i\delta} \frac{-\sqrt{3}t_{11} - t_{12} + 3t_{21} + \sqrt{3}t_{22}}{4} + e^{i\eta} e^{-i\gamma} \frac{\sqrt{3}t_{11} + t_{12} - 3 t_{21} -\sqrt{3}t_{22}}{4}                                                                                              \\
	                     & + e^{i\eta} e^{-i\delta}  \frac{-\sqrt{3}t_{11} + t_{12} - 3 t_{21} + \sqrt{3}t_{22}}{4} + e^{-i\eta} e^{i\gamma} \frac{\sqrt{3}t_{11} - t_{12} + 3 t_{21} - \sqrt{3}t_{22}}{4}                                                                                         \\
	                     & + t_{12}\left( e^{ik_x a} - e^{-ik_x a} \right)                                                                                                                                                                                                                         \\
	=                    & e^{-i\eta} e^{i\delta} \frac{-\sqrt{3}t_{11} - 4t_{12} + \sqrt{3}t_{22}}{4} + e^{i\eta} e^{-i\gamma} \frac{\sqrt{3}t_{11} + 4t_{12} -\sqrt{3}t_{22}}{4}                                                                                                                 \\
	                     & + e^{i\eta} e^{-i\delta}  \frac{-\sqrt{3}t_{11} + 4t_{12} + \sqrt{3}t_{22}}{4} + e^{-i\eta} e^{i\gamma} \frac{\sqrt{3}t_{11} - 4t_{12} - \sqrt{3}t_{22}}{4} + t_{12}\left( e^{ik_x a} - e^{-ik_x a} \right)                                                             \\
	=                    & \dfrac{\sqrt{3} t_{11}}{4}\left( - e^{-i\eta} e^{i\delta} + e^{i\eta} e^{-i\gamma} - e^{i\eta} e^{-i\delta} + e^{-i\eta} e^{i\gamma} \right) + t_{12}\left( - e^{-i\eta} e^{i\delta} + e^{i\eta} e^{-i\gamma} + e^{i\eta} e^{-i\delta} - e^{-i\eta} e^{i\gamma} \right) \\
	                     & + \dfrac{\sqrt{3} t_{22}}{4}\left( e^{-i\eta} e^{i\delta} - e^{i\eta} e^{-i\gamma} + e^{i\eta} e^{-i\delta} - e^{-i\eta} e^{i\gamma} \right) + t_{12}\left( e^{ik_x a} - e^{-ik_x a} \right)                                                                            \\
	=                    & 2it_{12}\sin2\alpha + \dfrac{\sqrt{3} t_{11} }{4}4\left[ -\cos\eta\sin\alpha\sin\beta + \sin\eta\cos\alpha\sin\beta \right]                                                                                                                                             \\
	                     & - 4it_{12}(\cos\eta\sin\alpha\cos\beta - \sin\eta\cos\alpha\cos\beta) + \sqrt{3} t_{22}\left[ \cos\eta\sin\alpha\sin\beta - \sin\eta\cos\alpha\sin\beta \right]                                                                                                         \\
	\Rightarrow h_{12} = & 4it_{12}(\sin\alpha\cos\alpha - \cos\eta\sin\alpha\cos\beta + \sin\eta\cos\alpha\cos\beta )                                                                                                                                                                             \\
	                     & + \dfrac{\sqrt{3} (t_{22} - t_{11}) }{4}4\left[ \cos\eta\sin\alpha\sin\beta - \sin\eta\cos\alpha\sin\beta  \right]                                                                                                                                                      \\
\end{align*}

\clearpage
Vậy Hamiltonian:
\begin{align}
	H_{TB}^{NN}(\mathbf{k}) =
	\begin{pmatrix}
		\renewcommand{\arraystretch}{0.72}
		h_0   & h_1      & h_2    \\
		h_1^* & h_{11}   & h_{12} \\
		h_2^* & h_{12}^* & h_{22}
	\end{pmatrix}
\end{align}\label{eq:1}
Với:
\begin{align}
	h_0      & = 2 t_0 \left[ \cos (2\alpha) + 2\cos\eta \cos\alpha\cos\beta + 2\sin\eta\sin\alpha\cos\beta\right] +   \epsilon_1,             \\
	h_1=     & 2 i t_1 (\sin 2\alpha + \cos\eta\sin\alpha\cos\beta -    \sin\eta\cos\alpha\cos\beta )       \nonumber                          \\
	         & - 2\sqrt{3}t_2\left( \cos\eta\sin\alpha\sin\beta - \sin\eta\cos\alpha\sin\beta  \right),                                        \\
	h_2      & = 2t_2 (\cos 2\alpha - \cos\eta\cos\alpha\cos\beta - \sin\eta\sin\alpha\cos\beta)                                               \\
	         & + 2i\sqrt{3}t_1(\cos\eta\cos\alpha\sin\beta +  \sin\eta\sin\alpha\sin\beta),                                                    \\
	h_{11}=  & (t_{11} + 3t_{22}) \left[ \cos\eta\cos\alpha\cos\beta + \sin\eta\sin\alpha\cos\beta \right] + 2t_{11}\cos2\alpha + \epsilon_2 , \\
	h_{22}=  & (3t_{11} + t_{22}) \left[ \cos\eta\cos\alpha\cos\beta + \sin\eta\sin\alpha\cos\beta \right] + 2t_{22}\cos2\alpha + \epsilon_2,  \\
	h_{12} = & 4it_{12}(\sin\alpha\cos\alpha -\cos\eta\sin\alpha\cos\beta + \sin\eta\cos\alpha\cos\beta ) \nonumber                            \\
	         & + \sqrt{3} (t_{22} - t_{11})\left[ \cos\eta\sin\alpha\sin\beta - \sin\eta\cos\alpha\sin\beta \right],
\end{align}
\begin{equation}
	\begin{split}
		(\alpha,\beta) & = \left(\dfrac{1}{2} k_x a,\dfrac{\sqrt{3}}{2} k_y a \right), \\
		\eta           & = \dfrac{e}{\hbar}\dfrac{Ba^2\sqrt{3}}{8},                    \\
	\end{split}
\end{equation}
\begin{equation}
	\begin{split}
		t_0    & = E_{11}^{11}(\mathbf{R_1}); \quad t_1 = E_{11}^{12}(\mathbf{R_1}); \quad t_2 = E_{12}^{12}(\mathbf{R_1}); \quad \\
		t_{11} & = E_{11}^{22}(\mathbf{R_1}); \quad t_{12} = E_{12}^{22}(\mathbf{R_1}); \quad t_{22} = E_{22}^{22}(\mathbf{R_1});
	\end{split}
\end{equation}

\clearpage

\noindent $\ast$ \textbf{Hamiltonian Zeeman:}

Chọn các cơ sở:
\begin{align*}
	\ket*{\phi^1_1 ,\uparrow}    = \ket*{\phi^1_1}\chi_{+}\quad & ,\quad \ket*{\phi^2_1 ,\uparrow} = \ket*{\phi^2_1}\chi_+ \quad,\quad \ket*{\phi^2_2 ,\uparrow} = \ket*{\phi^2_2}\chi_+         \\
	\ket*{\phi^1_1 ,\downarrow}  =	\ket*{\phi^1_1 }\chi_- \quad  & , \quad 	\ket*{\phi^2_1 ,\downarrow} = \ket*{\phi^2_1 }\chi_- \quad, \quad \ket*{\phi^2_2 ,\downarrow} = \ket*{\phi^2_2 }\chi_- \\
	                                                            & \chi_{+} = \begin{pmatrix}
		                                                                         \renewcommand{\arraystretch}{0.4}
		                                                                         1 \\
		                                                                         0
	                                                                         \end{pmatrix} \quad,\quad
	\chi_{-} = \begin{pmatrix}
		           \renewcommand{\arraystretch}{0.4}
		           0 \\
		           1
	           \end{pmatrix}
\end{align*}
Trong đó $\ket*{\phi_\mu^j}$ là các hàm sóng không gian,$\rchi$ là các hàm spinor. \\
Do các hàm Spinor $\rchi$ chỉ tác động lên spin $\sigma_z$ và không tác động lên Hamiltonian nằm trong không gian Hilbert. Đồng thời Hamiltonian \hyperref[eq:1]{(1)} không có sự tách spin nên ta có thể viết thành:
\begin{align}
	H = H_{space} + H_{1/2} = \mathbb{1}_{2\times2} \otimes H_{TB}^{NN} + H_{Zeeman}
\end{align}
Nhờ vào tính trực giao của các hàm cơ sở $\ket*{\phi_\mu^j}$ và spinor $\rchi$, ta tính được:\\
$\ast 	H_{11}^{11(z)} \uparrow$
\begin{align*}
	H_{11}^{11(z)} \uparrow & = - \sum_{\mathbf{R}} e^{\frac{ie}{\hbar}\int_0^{\mathbf{R}}\mathbf{A}(\mathbf{r'})\cdot d\mathbf{r'}}e^{i\mathbf{k\cdot R}} \bra{\phi^{1}_{1} \left(\mathbf{r}\right), \uparrow} \boldsymbol{\mu}\cdot \mathbf{B} \ket{\phi^{1}_{1} \left(\mathbf{r - R}\right) ,\uparrow} \\
	                        & = - \sum_{\mathbf{R}} e^{\frac{ie}{\hbar}\int_0^{\mathbf{R}}\mathbf{A}(\mathbf{r'})\cdot d\mathbf{r'}}e^{i\mathbf{k\cdot R}} \bra{\phi^{1}_{1} \left(\mathbf{r}\right) ,\uparrow} \gamma \mathbf{B}\cdot\mathbf{S} \ket{\phi^{1}_{1} \left(\mathbf{r - R}\right) ,\uparrow} \\
	                        & = -\gamma B \sum_{\mathbf{R}} e^{\frac{ie}{\hbar}\int_0^{\mathbf{R}}\mathbf{A}(\mathbf{r'})\cdot d\mathbf{r'}}e^{i\mathbf{k\cdot R}} \bra{\phi^{1}_{1} \left(\mathbf{r}\right) ,\uparrow} S_z \ket{\phi^{1}_{1} \left(\mathbf{r-R}\right) ,\uparrow}                        \\
	                        & = -\gamma B \sum_{\mathbf{R}} e^{\frac{ie}{\hbar}\int_0^{\mathbf{R}}\mathbf{A}(\mathbf{r'})\cdot d\mathbf{r'}}e^{i\mathbf{k\cdot R}} \bra{\phi_{1}^{1}(\mathbf{r})} \ket{\phi_{1}^{1}(\mathbf{r-R}) }\bra{\uparrow}S_z\ket{\uparrow}                                        \\
	                        & = \f{ -\gamma B\hbar}{2} \sum_{\mathbf{R}} e^{\frac{ie}{\hbar}\int_0^{\mathbf{R}}\mathbf{A}(\mathbf{r'})\cdot d\mathbf{r'}} \delta_{11} 1                                                                                                                                   \\
	                        & = \f{ -\gamma B\hbar}{2} (e^{0} + e^{-\frac{ie}{\hbar}\frac{Ba^2\sqrt{3}}{8}} + e^{\frac{ie}{\hbar}\frac{Ba^2\sqrt{3}}{8}} + e^{0} +
	e^{\frac{ie}{\hbar}\frac{Ba^2\sqrt{3}}{8}}
	+ e^{-\frac{ie}{\hbar}\frac{Ba^2\sqrt{3}}{8}} )                                                                                                                                                                                                                                                       \\
	                        & = \f{-\gamma B\hbar}{2} (2 + 4\cos\eta)  = -\gamma B\hbar(1+2\cos\eta)
\end{align*}
$\ast 	H_{11}^{22(z)} \uparrow$
\begin{align*}
	H_{11}^{22(z)} \uparrow & = - \sum_{\mathbf{R}} e^{\frac{ie}{\hbar}\int_0^{\mathbf{R}}\mathbf{A}(\mathbf{r'})\cdot d\mathbf{r'}}e^{i\mathbf{k\cdot R}} \bra{\phi^{2}_{1} \left(\mathbf{r}\right), \uparrow} \boldsymbol{\mu}\cdot \mathbf{B} \ket{\phi^{2}_{1} \left(\mathbf{r - R}\right) ,\uparrow} \\
	                        & = \f{ -\gamma B\hbar}{2} (e^{0} + e^{-\frac{ie}{\hbar}\frac{Ba^2\sqrt{3}}{8}} + e^{\frac{ie}{\hbar}\frac{Ba^2\sqrt{3}}{8}} + e^{0} +
	e^{\frac{ie}{\hbar}\frac{Ba^2\sqrt{3}}{8}}
	+ e^{-\frac{ie}{\hbar}\frac{Ba^2\sqrt{3}}{8}} )                                                                                                                                                                                                                                                       \\
	                        & = \f{-\gamma B\hbar}{2} (2 + 4\cos\eta)  = -\gamma B\hbar(1+2\cos\eta)
\end{align*}
$\ast 	H_{22}^{22(z)} \uparrow$
\begin{align*}
	H_{22}^{22(z)} \uparrow & = - \sum_{\mathbf{R}} e^{\frac{ie}{\hbar}\int_0^{\mathbf{R}}\mathbf{A}(\mathbf{r'})\cdot d\mathbf{r'}}e^{i\mathbf{k\cdot R}} \bra{\phi^{2}_{2} \left(\mathbf{r}\right), \uparrow} \boldsymbol{\mu}\cdot \mathbf{B} \ket{\phi^{2}_{2} \left(\mathbf{r - R}\right) ,\uparrow} \\
	                        & = \f{ -\gamma B\hbar}{2} (e^{0} + e^{-\frac{ie}{\hbar}\frac{Ba^2\sqrt{3}}{8}} + e^{\frac{ie}{\hbar}\frac{Ba^2\sqrt{3}}{8}} + e^{0} +
	e^{\frac{ie}{\hbar}\frac{Ba^2\sqrt{3}}{8}}
	+ e^{-\frac{ie}{\hbar}\frac{Ba^2\sqrt{3}}{8}} )                                                                                                                                                                                                                                                       \\
	                        & = \f{-\gamma B\hbar}{2} (2 + 4\cos\eta)  = -\gamma B\hbar(1+2\cos\eta).
\end{align*}
$\ast 	H_{11}^{11(z)} \downarrow$
\begin{align*}
	H_{11}^{11(z)} \downarrow & = - \sum_{\mathbf{R}} e^{\frac{ie}{\hbar}\int_0^{\mathbf{R}}\mathbf{A}(\mathbf{r'})\cdot d\mathbf{r'}}e^{i\mathbf{k\cdot R}} \bra{\phi^{1}_{1} \left(\mathbf{r}\right), \downarrow} \boldsymbol{\mu}\cdot \mathbf{B} \ket{\phi^{1}_{1} \left(\mathbf{r - R}\right) ,\downarrow} \\
	                          & = - \sum_{\mathbf{R}} e^{\frac{ie}{\hbar}\int_0^{\mathbf{R}}\mathbf{A}(\mathbf{r'})\cdot d\mathbf{r'}}e^{i\mathbf{k\cdot R}} \bra{\phi^{1}_{1} \left(\mathbf{r}\right) ,\downarrow} \gamma \mathbf{B}\cdot\mathbf{S} \ket{\phi^{1}_{1} \left(\mathbf{r - R}\right) ,\downarrow} \\s
	                          & = -\gamma B \sum_{\mathbf{R}} e^{\frac{ie}{\hbar}\int_0^{\mathbf{R}}\mathbf{A}(\mathbf{r'})\cdot d\mathbf{r'}}e^{i\mathbf{k\cdot R}} \bra{\phi^{1}_{1} \left(\mathbf{r}\right) ,\downarrow} S_z \ket{\phi^{1}_{1} \left(\mathbf{r-R}\right) ,\downarrow}                        \\
	                          & = -\gamma B \sum_{\mathbf{R}} e^{\frac{ie}{\hbar}\int_0^{\mathbf{R}}\mathbf{A}(\mathbf{r'})\cdot d\mathbf{r'}}e^{i\mathbf{k\cdot R}} \bra{\phi_{1}^{1}(\mathbf{r})} \ket{\phi_{1}^{1}(\mathbf{r-R}) }\bra{\downarrow}S_z\ket{\downarrow}                                        \\
	                          & = \f{ -\gamma B\hbar}{2} \sum_{\mathbf{R}} e^{\frac{ie}{\hbar}\int_0^{\mathbf{R}}\mathbf{A}(\mathbf{r'})\cdot d\mathbf{r'}} \delta_{11} 1                                                                                                                                       \\
	                          & = \f{ -\gamma B\hbar}{2} (e^{0} + e^{-\frac{ie}{\hbar}\frac{Ba^2\sqrt{3}}{8}} + e^{\frac{ie}{\hbar}\frac{Ba^2\sqrt{3}}{8}} + e^{0} +
	e^{\frac{ie}{\hbar}\frac{Ba^2\sqrt{3}}{8}}
	+ e^{-\frac{ie}{\hbar}\frac{Ba^2\sqrt{3}}{8}} )                                                                                                                                                                                                                                                             \\
	                          & = \f{-\gamma B\hbar}{2} (2 + 4\cos\eta)  = \gamma B\hbar(1+2\cos\eta)
\end{align*}
$\ast H_{11}^{22(z)} \downarrow$
\begin{align*}
	H_{11}^{22(z)} \downarrow & = - \sum_{\mathbf{R}} e^{\frac{ie}{\hbar}\int_0^{\mathbf{R}}\mathbf{A}(\mathbf{r'})\cdot d\mathbf{r'}}e^{i\mathbf{k\cdot R}} \bra{\phi^{2}_{1} \left(\mathbf{r}\right), \downarrow} \boldsymbol{\mu}\cdot \mathbf{B} \ket{\phi^{2}_{1} \left(\mathbf{r - R}\right) ,\downarrow} \\
	                          & = \f{ -\gamma B\hbar}{2} (e^{0} + e^{-\frac{ie}{\hbar}\frac{Ba^2\sqrt{3}}{8}} + e^{\frac{ie}{\hbar}\frac{Ba^2\sqrt{3}}{8}} + e^{0} +
	e^{\frac{ie}{\hbar}\frac{Ba^2\sqrt{3}}{8}}
	+ e^{-\frac{ie}{\hbar}\frac{Ba^2\sqrt{3}}{8}} )                                                                                                                                                                                                                                                             \\
	                          & = \f{-\gamma B\hbar}{2} (2 + 4\cos\eta)  = \gamma B\hbar(1+2\cos\eta)
\end{align*}
$\ast 	H_{22}^{22(z)} \downarrow$
\begin{align*}
	H_{22}^{22(z)} \downarrow & = - \sum_{\mathbf{R}} e^{\frac{ie}{\hbar}\int_0^{\mathbf{R}}\mathbf{A}(\mathbf{r'})\cdot d\mathbf{r'}}e^{i\mathbf{k\cdot R}} \bra{\phi^{2}_{2} \left(\mathbf{r}\right), \downarrow} \boldsymbol{\mu}\cdot \mathbf{B} \ket{\phi^{2}_{2} \left(\mathbf{r - R}\right) ,\downarrow} \\
	                          & = \f{ -\gamma B\hbar}{2} (e^{0} + e^{-\frac{ie}{\hbar}\frac{Ba^2\sqrt{3}}{8}} + e^{\frac{ie}{\hbar}\frac{Ba^2\sqrt{3}}{8}} + e^{0} +
	e^{\frac{ie}{\hbar}\frac{Ba^2\sqrt{3}}{8}}
	+ e^{-\frac{ie}{\hbar}\frac{Ba^2\sqrt{3}}{8}} )                                                                                                                                                                                                                                                             \\
	                          & = \f{-\gamma B\hbar}{2} (2 + 4\cos\eta)  = \gamma B\hbar(1+2\cos\eta)
\end{align*}
với $\gamma$ = $-\dfrac{e}{m}$

Hamiltonian cho thành phần Zeeman:
\begin{align*}
	H_{Zeeman} = \f{e\hbar B}{m}(1+\cos\eta)
	\begin{pmatrix}
		1 & 0 & 0 & 0  & 0  & 0  \\
		0 & 1 & 0 & 0  & 0  & 0  \\
		0 & 0 & 1 & 0  & 0  & 0  \\
		0 & 0 & 0 & -1 & 0  & 0  \\
		0 & 0 & 0 & 0  & -1 & 0  \\
		0 & 0 & 0 & 0  & 0  & -1
	\end{pmatrix}
\end{align*}
Ta có thể xây dựng Hamiltonian thành:
\begin{align}
	H & =
	\begin{pmatrix}
		1 & 0 \\
		0 & 1
	\end{pmatrix}\bigotimes
	\begin{pmatrix}
		h_0   & h_1      & h_2    \\
		h_1^* & h_{11}   & h_{12} \\
		h_2^* & h_{12}^* & h_{22}
	\end{pmatrix}
	+
	H_{Zeeman}
	\nonumber \\
	  & =
	\begin{pmatrix}
		h_0   & h_1      & h_2    & 0       & 0        & 0      \\
		h_1^* & h_{11}   & h_{12} & 0       & 0        & 0      \\
		h_2^* & h_{12}^* & h_{22} & 0       & 0        & 0      \\
		0     & 0        & 0      & h_{0}   & h_1      & h_{2}  \\
		0     & 0        & 0      & h_1^*   & h_{11}   & h_{12} \\
		0     & 0        & 0      & h_{2}^* & h_{12}^* & h_{22} \\
	\end{pmatrix}
	+
	\f{e\hbar B}{m}(1+\cos\eta)
	\begin{pmatrix}
		1 & 0 & 0 & 0  & 0  & 0  \\
		0 & 1 & 0 & 0  & 0  & 0  \\
		0 & 0 & 1 & 0  & 0  & 0  \\
		0 & 0 & 0 & -1 & 0  & 0  \\
		0 & 0 & 0 & 0  & -1 & 0  \\
		0 & 0 & 0 & 0  & 0  & -1 \\
	\end{pmatrix} \nonumber
\end{align}
\clearpage
Chéo hóa Hamiltonian, ta có phương trình hàm riêng trị riêng:
\begin{align*}
	H_{TB}^{NN}(\mathbf{k}) f & = \lambda f \\
	\begin{pmatrix}
		\renewcommand{\arraystretch}{0.72}
		h_0   & h_1      & h_2    \\
		h_1^* & h_{11}   & h_{12} \\
		h_2^* & h_{12}^* & h_{22}
	\end{pmatrix}f
	                          & = \lambda
	\begin{pmatrix}
		\renewcommand{\arraystretch}{0.72}
		1 & 0 & 0 \\
		0 & 1 & 0 \\
		0 & 0 & 1 \\
	\end{pmatrix}f
\end{align*}
\begin{align*}
	\Rightarrow
	\begin{pmatrix}
		\renewcommand{\arraystretch}{0.72}
		h_0 - \lambda & h_1              & h_2              \\
		h_1^*         & h_{11} - \lambda & h_{12}           \\
		h_2^*         & h_{12}^*         & h_{22} - \lambda
	\end{pmatrix}f
	 & = 0
\end{align*}

Để phương trình có nghiệm không tầm thường: $\Leftrightarrow$
$
	\begin{vmatrix}
		\renewcommand{\arraystretch}{0.5}
		h_0 - \lambda & h_1              & h_2              \\
		h_1^*         & h_{11} - \lambda & h_{12}           \\
		h_2^*         & h_{12}^*         & h_{22} - \lambda
	\end{vmatrix} = 0
$
\begin{equation*}
	h_{1}^{}\left[h_{12}^{}h_2^* - h_1^*(h_{22}^{} - \lambda)\right] + h_2^{}\left[h_{12}^* h_1^* - h_2^*(h_{11}^{} - \lambda) \right] + (h_0^{} - \lambda)\left[(h_{11}^{} - \lambda)(h_{22}^{} - \lambda) - h_{12}^{}h_{12}^*\right] = 0
\end{equation*}
\begin{align*}
	\Leftrightarrow & h_{1}^{} h_{12}^{} h_2^{*} - h_{1}^{} h_{1}^{*} h_{22}^{} + h_{1}^{} h_{1}^{*} \lambda + h_{2}^{} h_{12}^{*} h_1^{*} - h_{2}^{} h_{2}^{*} h_{11} + h_{2}^{} h_{2}^{*} \lambda \\
	                & + (h_{0}^{} - \lambda) (h_{11}^{} - \lambda) (h_{22}^{} - \lambda) - h_{0}^{} h_{12}^{} h_{12}^{*} + h_{12}^{} h_{12}^{*} \lambda = 0
\end{align*}

\clearpage
\textbf{Two bands} $\mathbf{k\cdot p}$\;\textbf{model}

Nếu bỏ qua tương tác Coulomb giữa các điện tử, Hamiltonian của hệ nhiều điện tử đơn giản là tổng các Hamiltonian một điện tử:
\begin{align}
	H = \sum_i H_{1e}(\mathbf{r}_i) = \sum_i \left(-\f{\hbar^2 \nabla_i^2}{2m_0} + V_0(\mathbf{r}_i) \right).
\end{align}
Hàm sóng trong mạng tinh thể thỏa định lý Bloch:
\begin{align}
	\ket{\psi_{m\mathbf{k}}(\mathbf{r})} = e^{i\mathbf{k \cdot r}} \ket{u_{m\mathbf{k}}(\mathbf{r})}.
\end{align}
Thay (13) vào (12), ta được phương trình Schr\"{o}dinger cho mạng tinh thể tuần hoàn theo $u_{m\mathbf{k}}$:
\begin{align}
	\left[ \f{p^2}{2m_0} + V_0(\mathbf{r}) + \f{\hbar^2 k^2}{2m_0} + \f{\hbar}{m_0} \mathbf{k\cdot p} \right] \ket{u_{m\mathbf{k}}(\mathbf{r})} = E_{m\mathbf{k}} \ket{u_{m\mathbf{k}}(\mathbf{r})}.
\end{align}
Giả định rằng chúng ta đã biết trị riêng năng lượng và trạng thái riêng tại một điểm $k_0$ trong vùng Brillouin. Để giải phương trình (14), ta có thể khai triển hàm riêng $\ket{u_{m\mathbf{k}}(\mathbf{r})}$ qua một tập hợp các hàm cơ sở trực chuẩn, đầy đủ $\{\ket{u_n}\}$:
\begin{align}
	\ket{u_{m\mathbf{k}}(\mathbf{r})} = \sum_n a_{m\mathbf{k}}^{n} \ket{u_{n}(\mathbf{r})}.
\end{align}
Thay (15) vào (14) và nhân trái với $\bra{u_{n}(\mathbf{r})}$, ta được:
\begin{align}
	\sum_{n'} H_{nn'}(\mathbf{k}) a_{m\mathbf{k}}^{n'} = E_{m\mathbf{k}} a_{m\mathbf{k}}^{n}  ,
\end{align}
trong đó
\begin{align}
	H_{nn'}(\mathbf{k})
	 & = \left(E_n^0 + \f{\hbar^2 k^2}{2m_0} \right)\delta_{nn'} + \f{\hbar}{m_0} \mathbf{k}\cdot \bra{u_{n}}\mathbf{p}\ket{u_{n'}}.
\end{align}
$H_{\mathbf{k\cdot p}} = \f{\hbar}{m_0} \mathbf{k}\cdot \bra{u_{n}}\mathbf{p}\ket{u_{n'}}$

Ta đi khai triển nhiễu loạn cho $H_{\mathbf{k\cdot p}}$ tới bậc 3 lân cận $\mathbf{k}$:
\begin{align}
	H_{\mathbf{k\cdot p}} = \f{\hbar}{m_0} \mathbf{k}\cdot \bra{u_{n}}\mathbf{p}\ket{u_{n'}},
\end{align}
$\bra{u_{n}}\mathbf{p}\ket{u_{n}} =0$. \\
Hamiltonian cho tập hợp con $A = \{ \ket{u_m}\}$:
\begin{align}
	\breve{H}_A = H^{(0)} + H^{(1)} + H^{(2)} + H^{(3)}
\end{align}
trong đó
\begin{align*}
	H^{(1)}_{ii'} & = H^{'}_{ii'},                                                                                                                                                                                              \\
	H^{(2)}_{ii'} & = \f{1}{2} \sum_l H^{'}_{in} H^{'}_{ni'} \left[ \f{1}{E_i - E_n} + \f{1}{E_{i'} - E_n} \right],                                                                                                             \\
	H^{(3)}_{ii'} & = -\f{1}{2} \sum_{n,i''} \left[ \f{ H^{'}_{in}H^{'}_{ni''}H^{'}_{i'' i'} }{ (E_{i'} - E_{n})(E_{i''} - E_{n})} +  \f{ H^{'}_{ii''}H^{'}_{i''n}H^{'}_{n i'} }{ (E_{i} - E_{n})(E_{i''} - E_{n})  } \right] , \\
	              & + \f{1}{2} \sum_{n,n'}^{} H^{'}_{in}H^{'}_{nn'}H^{'}_{n'i'}\left[ \f{1}{ (E_{i} - E_{n})(E_{i} - E_{n'})} + \f{1}{ (E_{i'} - E_{n})(E_{i'} - E_{n'})} \right],
\end{align*}
với $i,i',i'' \in A$ và $n,n' \in B$ $(i = \pm 2,0)$ và $(n = \pm 1)$. Ứng với đó là $d_{\pm 2} = \frac{1}{\sqrt{2}}(d_{x^2-y^2} \pm i d_{xy}), d_0 = d_{z^2}, d_{\pm 1} = \frac{1}{\sqrt{2}}(d_{xz} \pm i d_{yz})$.\\
Do đó ta chọn các cơ sở:
\begin{align*}
	 & \ket{\psi_{c}^{\tau}} = \ket{d_{z^2}} \equiv \ket{\phi^1_1}                                                                                                                      \\
	 & \ket{\mathcal{\psi}_{v}^{\tau}} = \f{1}{\sqrt{2}} \left(\ket{d_{x^2-y^2}} + i\tau \ket{d_{xy}}\right)  \equiv \f{1}{\sqrt{2}} \left(\ket{\phi^2_2} + i\tau \ket{\phi^2_1}\right)
\end{align*}

\begin{align}
	\bra{u_c,\pm \mathbf{K}} p_x \ket{u_v,\pm \mathbf{K}} = \pm i \bra{u_c,\pm \mathbf{K}} p_y \ket{u_v,\pm \mathbf{K}}
\end{align}
Thành phần ma trận của Hamiltonian không nhiễu loạn $H^0$:

\begin{align*}
	H_{\mu\mu'}^{}(\mathbf{k},\tau)
	 & = \sum_{R} e^{i\mathbf{k\cdot R}} E_{\mu\mu'}^{}(\mathbf{R})                                                                             \\
	 & = \sum_{R} e^{i\mathbf{k\cdot R}}\bra{\psi_{\mu}^{\mathfrak{\tau}}(\mathbf{r})}\hat{H}\ket{\mathcal{\psi}_{\mu'}^{\tau}(\mathbf{r - R})}
\end{align*}

$\ast H_{cc}(\mathbf{k},\tau)$
\begin{align*}
	H_{cc}(\mathbf{k},\tau)
	 & = \sum_{R} e^{i\mathbf{k\cdot R}}\bra{\psi_{c}^{\mathfrak{\tau}}(\mathbf{r})}\hat{H}\ket{\mathcal{\psi}_{c}^{\tau}(\mathbf{r - R})} \\
	 & = h_0                                                                                                                               \\
	 & = 2 t_0 \left(\cos2\alpha + 2\cos\alpha \cos\beta\right) + \epsilon_1
\end{align*}

$\ast H_{cv}(\mathbf{k},\tau)$
\begin{align*}
	H_{cv}(\mathbf{k},\tau)
	 & = \sum_{R} e^{i\mathbf{k\cdot R}}\bra{\psi_{c}^{\mathfrak{\tau}}(\mathbf{r})}\hat{H}\ket{\mathcal{\psi}_{v}^{\tau}(\mathbf{r - R})}                                                                 \\
	 & = \f{1}{\sqrt{2}}\sum_{R} e^{i\mathbf{k\cdot R}}\left[ \bra{\phi^1_1(\mathbf{r})} \hat{H} \ket{\phi^2_2(\mathbf{r-R})} + i\tau\bra{\phi^1_1(\mathbf{r})}\hat{H}\ket{\phi^2_1(\mathbf{r-R})} \right] \\
	 & = \f{1}{\sqrt{2}} (h_2 + i\tau h_1)                                                                                                                                                                 \\
	 & = \frac{1}{\sqrt{2}} \biggl[ 2t_2 (\cos 2\alpha -\cos\alpha\cos\beta ) + 2i\sqrt{3}t_1\cos\alpha\sin\beta                                                                                           \\
	 & + i\tau \left(-2\sqrt{3} t_2 \sin\alpha \sin\beta + 2i t_1(\sin2\alpha + \sin\alpha \cos\beta )\right) \biggr]
\end{align*}

$\ast H_{vc}(\mathbf{k},\tau)$
\begin{align*}
	H_{vc}(\mathbf{k},\tau)
	 & = \sum_{R} e^{i\mathbf{k\cdot R}}\bra{\psi_{v}^{\mathfrak{\tau}}(\mathbf{r})}\hat{H}\ket{\mathcal{\psi}_{c}^{\tau}(\mathbf{r - R})}                                                                 \\
	 & = \f{1}{\sqrt{2}}\sum_{R} e^{i\mathbf{k\cdot R}}\left[ \bra{\phi^2_2(\mathbf{r})} \hat{H} \ket{\phi^1_1(\mathbf{r-R})} - i\tau\bra{\phi^2_1(\mathbf{r})}\hat{H}\ket{\phi^1_1(\mathbf{r-R})} \right] \\
	 & = \f{1}{\sqrt{2}} (h_2^{\ast} - i\tau h_1^{\ast})                                                                                                                                                   \\
	 & = \frac{1}{\sqrt{2}} \biggl[ 2t_2 (\cos 2\alpha -\cos\alpha\cos\beta ) - 2i\sqrt{3}t_1\cos\alpha\sin\beta                                                                                           \\
	 & - i\tau \left(-2\sqrt{3} t_2 \sin\alpha \sin\beta - 2i t_1(\sin2\alpha + \sin\alpha \cos\beta )\right) \biggr]
\end{align*}

$\ast H_{vv}(\mathbf{k},\tau)$
\begin{align*}
	H_{vv}(\mathbf{k},\tau)
	 & = \sum_{R} e^{i\mathbf{k\cdot R}}\bra{\psi_{v}^{\mathfrak{\tau}}(\mathbf{r})}\hat{H}\ket{\mathcal{\psi}_{v}^{\tau}(\mathbf{r - R})}                                                           \\
	 & = \f{1}{2}\sum_{R} e^{i\mathbf{k\cdot R}} \biggl[ \bra{\phi^2_2(\mathbf{r})} \hat{H} \ket{\phi^2_2(\mathbf{r-R})} + i\tau\bra{\phi^2_2(\mathbf{r})}\hat{H}\ket{\phi^2_1(\mathbf{r-R})}        \\
	 & - i\tau \bra{\phi^2_1(\mathbf{r})}\hat{H}\ket{\phi^2_2(\mathbf{r-R})} + \tau^2 \bra{\phi^2_1(\mathbf{r})}\hat{H}\ket{\phi^2_1(\mathbf{r-R})} \biggr]                                          \\
	 & = \f{1}{2} \biggl( h_{22} + i\tau h_{12}^{\ast} - i\tau h_{12} + \tau h_{11}   \biggr)                                                                                                        \\
	 & = \f{1}{2} \biggl[ 2 t_{22} \cos2\alpha  + (3t_{11} + t_{22})\cos\alpha \cos\beta + \tau^2 \left(2t_{11}\cos2\alpha + \left (t_{11} + 3t_{22}\right )\cos\alpha\cos\beta + 2\epsilon_2\right) \\
	 & - i\tau (8i t_{12}\sin\alpha(\cos\alpha - \cos\beta)) \biggr]
\end{align*}
Tại $\pm K$ valley

\begin{align*}
	 & \mathbf{k}  = (k_x,k_y) = \left(\tau\f{4\pi}{3a},0\right)                   \\
	 & (\alpha,\beta) = \left(\dfrac{1}{2} k_x a,\dfrac{\sqrt{3}}{2} k_y a \right)
\end{align*}
với $\tau = \pm 1$
\begin{align*}
	H_{cc}(\mathbf{k},\tau)
	 & =  -3t_0  + \epsilon_1
\end{align*}
\begin{align*}
	H_{cv}(\mathbf{k},\tau)
	 & = 0
\end{align*}
\begin{align*}
	H_{vc}(\mathbf{k},\tau)
	 & = 0
\end{align*}
\begin{align*}
	H_{vv}(\mathbf{k},\tau)
	 & =  \epsilon_2 - \f{3}{2} (t_{11} + t_{22}) + \tau 3\sqrt{3}t_{12}
\end{align*}

$\ast H^{(1)}_{mm'}$

\begin{align*}
	H^{(1)}_{cc}
	 & = H^{'}_{0,0} = 0
\end{align*}
\begin{align*}
	H^{(1)}_{vv}
	 & = H^{'}_{2,2} = 0
\end{align*}
\begin{align*}
	H^{(1)}_{vc}
	 & = H^{'}_{vc}      \\
	 & = H_{0,2}^{'\ast}
\end{align*}
\begin{align*}
	H^{(1)}_{cv}
	 & = H^{'}_{0,2}                                                                                   \\
	 & = \f{\hbar}{m_0} \mathbf{k}\cdot \bra{u_{c}}\mathbf{p}\ket{u_{v}}                               \\
	 & = \f{\hbar}{m_0} \left(k_x \bra{u_{c}} p_x \ket{u_{v}} + k_y \bra{u_{c}} p_y \ket{u_{v}}\right) \\
	 & = \f{\hbar}{m_0} \left( k_x \bra{u_c} p_x \ket{u_v} -i k_y \bra{u_c} p_x \ket{u_v} \right)      \\
	 & = \f{\hbar}{m_0} \left( k_x \bra{u_c} p_x \ket{u_v} - i k_y \bra{u_c} p_x \ket{u_v} \right)     \\
	 & = \f{a\hbar}{a m_0} \left( k_x - i k_y \right)\bra{u_c} p_x \ket{u_v}                           \\
	 & = at \left( k_x - i k_y \right)
\end{align*}
đặt $t = \f{\hbar}{a m_0}\bra{u_c} p_x \ket{u_v}$
(ta nhân thêm $a$ và chia cho $a$ ở mẫu để không bị vi phạm thứ nguyên).\\
%Ta đã sử dụng mô hình TB ba dải cho đơn lớp $MX_2$ bằng cách sử dụng $d_{z^2},d_{xy},d_{x^2-y^2}$ orbitals là các hàm sóng cơ sở.

Ta sử dụng định nghĩa mới là $\mathbf{k = q + K}$, và viết lại phương trình (18) dưới dạng:

\begin{align}
	H_{\mathbf{k\cdot p}} = \f{1}{2} \f{\hbar}{m_0} ( q_{+} \hat{p}_{-} + q_{-} \hat{p}_{+}) = H_{\mathbf{k\cdot p}}^{-} + H_{\mathbf{k\cdot p}}^{+},
\end{align}
với $q_{\pm} = q_{x} \pm i q_y $, $\hat{p}_{\pm} = \hat{p}_{x} \pm i \hat{p}_y $

\begin{table}[h!]
	\centering
	\begin{tabular}{c  c  c}
		\hline
		\hline
		irrep       & Basics funtioncs                           & Band \\ [0.6ex]
		\hline
		$A_{1}^{'}$ & $\ket{\Psi_{2,0}}$                         & VB   \\
		$E_{}^{'}$  & $\{ \ket{\Psi_{2,2}},\ket{\Psi_{2,-2}} \}$ & VB-3 \\
		$E_{}^{''}$ & $\{ \ket{\Psi_{2,1}},\ket{\Psi_{2,-1}} \}$ & VB-1 \\
		\hline
	\end{tabular}
	\caption{Cơ sở cho cho biểu diển bất khả quy của nhóm $D_{3h}$ tại điểm $\Gamma$}.
\end{table}

Sử dụng toán tử quay $C_3$ tác dụng lên $p_\pm$:
\begin{align*}
	\renewcommand{\arraystretch}{0.75}
	C_{3} \hat{p}_{+} & =
	\begin{pmatrix}
		-\f{1}{2}       & -\f{\sqrt{3}}{2} \\
		\f{\sqrt{3}}{2} & -\f{1}{2}        \\
	\end{pmatrix}
	\begin{pmatrix}
		p_x \\
		p_y
	\end{pmatrix}
	= \begin{pmatrix}
		  p'_x \\
		  p'_y
	  \end{pmatrix}                                                                                                             \\
	                  & = \left( -\f{1}{2} p_x - \f{\sqrt{3}}{2} p_y \right) + i \left( \f{\sqrt{3}}{2}p_x - \f{1}{2}p_y \right) \\
	                  & = e^{-\frac{2\pi i}{3}} p_{+}
\end{align*}
Tương tự cho $p_{-}$, ta có:
\begin{align}
	C_{3} \hat{p}_{\pm} = e^{\mp\frac{2\pi i}{3}} p_{\pm}
\end{align}
Sử dụng kết quả (22), ta tính được nhiễu loạn bậc 1 như sau:\\
$\ast H^{(1)}_{mm'}$

\begin{align*}
	H^{(1)}_{cc}
	 & = H^{'}_{0,0} = 0
\end{align*}
\begin{align*}
	H^{(1)}_{vv}
	 & = H^{'}_{2,2} = 0
\end{align*}
\begin{align*}
	H^{(1)}_{vc}
	 & = H^{'}_{vc}      \\
	 & = H_{0,2}^{'\ast}
\end{align*}
\begin{align*}
	H^{(1)}_{cv}
	 & = H^{'}_{0,2}                                                                                                                                                                                                                   \\
	 & = \f{1}{2}\f{\hbar}{m_0}  \left( q_{+}\bra{0}\hat{p}_{-}\ket{2} + q_{-}\bra{0}\hat{p}_{+}\ket{2}  \right)                                                                                                                       \\
	 & = \f{1}{2}\f{\hbar}{m_0}  \left( q_{+}\bra{0}C_{3}^{\dagger}C_{3}\hat{p}_{-}C_{3}^{\dagger}C_{3}\ket{2} + q_{-}\bra{0}C_{3}^{\dagger}C_{3}\hat{p}_{+}C_{3}^{\dagger}C_{3}\ket{2}  \right)                                       \\
	 & = \f{1}{2}\f{\hbar}{m_0} \left( e^{\frac{2i\pi}{3}} e^{\frac{2i\pi}{3}}q_{+} \cancelto{0}{\bra{0} p_{-}\ket{2}} + e^{-\frac{2i\pi}{3}} e^{\frac{2i\pi}{3}}q_{-}\bra{0}\hat{p}_{+}\ket{2}  \right)(\text{để dấu ``$=$'' xảy ra}) \\
	 & = \f{a\hbar}{a m_0} q_{-}\bra{u_c} \hat{p}_{+} \ket{u_v}                                                                                                                                                                        \\
	 & = at q_{-}
\end{align*}

$\ast H^{(2)}_{mm'}$
\begin{align*}
	H_{0,0}^{(2)}
	       & = \f{1}{2} \sum_{l} H_{0,l}^{'}H_{l,0}^{'}\left[ \f{1}{ E_{0} - E_{l}} + \f{1}{ E_{0} - E_{l} } \right]                                                                                                                                                                                      \\
	       & = \f{1}{2} H_{0,-1}^{'}H_{-1,0}^{'}\left[ \f{1}{ E_{0} - E_{-1}} + \f{1}{ E_{0} - E_{-1} } \right] +                                                                                            \f{1}{2} H_{0,1}^{'}H_{1,0}^{'}\left[ \f{1}{ E_{0} - E_{1}} + \f{1}{ E_{0} - E_{1} } \right] \\
	       & = \f{1}{4}\f{\hbar}{m_0} \left( q_{+}\bra{0}\hat{p}_{-}\ket{-1} + q_{-}\bra{0}\hat{p}_{+}\ket{-1}\right)\left( q_{+}\bra{-1}\hat{p}_{-}\ket{0} + q_{-}\bra{-1}\hat{p}_{+}\ket{0}\right)\left[ \f{1}{ E_{0} - E_{-1}} + \f{1}{ E_{0} - E_{-1} } \right]                                       \\
	+      & \f{1}{4}\f{\hbar}{m_0} \left( q_{+}\bra{0}\hat{p}_{-}\ket{1} + q_{-}\bra{0}\hat{p}_{+}\ket{1}\right)\left( q_{+}\bra{1}\hat{p}_{-}\ket{0} + q_{-}\bra{1}\hat{p}_{+}\ket{0}\right)\left[ \f{1}{ E_{0} - E_{1}} + \f{1}{ E_{0} - E_{1} } \right]                                               \\
	       & = \f{1}{4}\f{\hbar}{m_0} \left( q_{+}\bra{0}C_{3}^{\dagger}C_{3}\hat{p}_{-}C_{3}^{\dagger}C_{3}\ket{-1} + q_{-}\bra{0}C_{3}^{\dagger}C_{3}\hat{p}_{+}C_{3}^{\dagger}C_{3}\ket{-1}\right)                                                                                                     \\
	\times & \left( q_{+}\bra{-1}C_{3}^{\dagger}C_{3}\hat{p}_{-}C_{3}^{\dagger}C_{3}\ket{0} + q_{-}\bra{-1}C_{3}^{\dagger}C_{3}\hat{p}_{+}C_{3}^{\dagger}C_{3}\ket{0}\right) \left[ \f{1}{ E_{0} - E_{-1}} + \f{1}{ E_{0} - E_{-1} } \right]                                                              \\
	+      & \f{1}{4}\f{\hbar}{m_0} \left( q_{+}\bra{0}C_{3}^{\dagger}C_{3}\hat{p}_{-}C_{3}^{\dagger}C_{3}\ket{1} + q_{-}\bra{0}C_{3}^{\dagger}C_{3}\hat{p}_{+}C_{3}^{\dagger}C_{3}\ket{1}\right)                                                                                                         \\
	\times & \left( q_{+}\bra{1}C_{3}^{\dagger}C_{3}\hat{p}_{-}C_{3}^{\dagger}C_{3}\ket{0} + q_{-}\bra{1}C_{3}^{\dagger}C_{3}\hat{p}_{+}C_{3}^{\dagger}C_{3}\ket{0}\right)\left[ \f{1}{ E_{0} - E_{1}} + \f{1}{ E_{0} - E_{1} } \right]                                                                   \\
	       & = \f{1}{4}\f{\hbar}{m_0} \left( q_{+} e^{\frac{2i\pi}{3}}e^{-\frac{2i\pi}{3}} \bra{0}\hat{p}_{-}\ket{-1} + q_{-}e^{-\frac{2i\pi}{3}}e^{-\frac{2i\pi}{3}}\cancelto{0}{\bra{0}\hat{p}_{+}\ket{-1}} \right)                                                                                     \\
	\times & \left( q_{+} e^{\frac{2i\pi}{3}}e^{\frac{2i\pi}{3}} \cancelto{0}{\bra{-1}\hat{p}_{-}\ket{0}} + q_{-}e^{-\frac{2i\pi}{3}}e^{\frac{2i\pi}{3}}\bra{-1}\hat{p}_{+}\ket{0}\right)\left[ \f{1}{ E_{0} - E_{-1}} + \f{1}{ E_{0} - E_{-1} } \right]                                                  \\
	+      & \f{1}{4}\f{\hbar}{m_0} \left( q_{+} e^{\frac{2i\pi}{3}}e^{-\frac{2i\pi}{3}} \bra{0}\hat{p}_{-}\ket{1} + q_{-}e^{-\frac{2i\pi}{3}}e^{-\frac{2i\pi}{3}}\cancelto{0}{\bra{0}\hat{p}_{+}\ket{1}} \right)                                                                                         \\
	\times & \left( q_{+} e^{\frac{2i\pi}{3}}e^{\frac{2i\pi}{3}} \cancelto{0}{\bra{1}\hat{p}_{-}\ket{0}} + q_{-}e^{-\frac{2i\pi}{3}}e^{\frac{2i\pi}{3}}\bra{1}\hat{p}_{+}\ket{0}\right)\left[ \f{1}{ E_{0} - E_{1}} + \f{1}{ E_{0} - E_{1} } \right] (\text{để dấu ``$=$'' xảy ra})
\end{align*}
Vậy, rút gọn những thành phần ma trận không cần thiết ta được:
\begin{align*}
	H_{0,0}^{(2)}
	  & =\f{1}{4}\f{\hbar}{m_0} \left( q_{+} \bra{0}\hat{p}_{-}\ket{-1} \right) \left( q_{-}\bra{-1}\hat{p}_{+}\ket{0} \right)\left[ \f{2}{ E_{0} - E_{-1}} \right] \\
	+ & \f{1}{4}\f{\hbar}{m_0} \left( q_{+} \bra{0}\hat{p}_{-}\ket{1} \right) \left( q_{-}\bra{1}\hat{p}_{+}\ket{0} \right)\left[ \f{2}{ E_{0} - E_{1}} \right]     \\
	  & = \f{1}{2} \f{\hbar}{m_0}q^2 \left[  \bra{0}\hat{p}_{-}\ket{-1}\bra{-1}\hat{p}_{+}\ket{0}\f{1}{ E_{0} - E_{-1}} + \bra{0}\hat{p}_{-}\ket{1}\bra{1}\hat{p}_{+}\ket{0}\f{1}{ E_{0} - E_{1}}  \right] \\
	  & = q^2 a^2 \underbracket{\f{1}{2} \f{\hbar}{a^2 m_0} \left[  \bra{0}\hat{p}_{-}\ket{-1}\bra{-1}\hat{p}_{+}\ket{0}\f{1}{ E_{0} - E_{-1}} + \bra{0}\hat{p}_{-}\ket{1}\bra{1}\hat{p}_{+}\ket{0}\f{1}{ E_{0} - E_{1}}  \right]}_{\gamma_1} \\
	  & = q^2 a^2 {\gamma_1}
\end{align*}
với $q^2 = q_x^2 + q_y^2$.
\begin{align*}
	H_{2,-1}^{(2)}
	 & = \f{1}{2} \sum_{l} H_{2,l}^{'}H_{l,-2}^{'}\left[ \f{1}{ E_{2} - E_{l}} + \f{1}{ E_{-2} - E_{l} } \right]                                                                                                      \\
	 & = \f{1}{2} H_{2,-1}^{'}H_{-1,-2}^{'}\left[ \f{1}{ E_{2} - E_{-1}} + \f{1}{ E_{-2} - E_{-1} } \right] + \f{1}{2} H_{2,1}^{'}H_{1,-2}^{'} \left[ \f{1}{ E_{2} - E_{1}} + \f{1}{ E_{-2} - E_{1} } \right]         \\	
	 & = \f{1}{2} H_{2,-1}^{'}H_{-1,-2}^{'}\left[ \f{1}{ E_{2} - E_{-1}} + \f{1}{ E_{-2} - E_{-1} } \right] + \f{1}{2} H_{2,1}^{'}H_{1,-2}^{'} \left[ \f{1}{ E_{2} - E_{1}} + \f{1}{ E_{-2} - E_{1} } \right]         \\
\end{align*}























\newpage
\begin{table}[h!]
	\centering
	\begin{tabular}{c  c  c}
		\hline
		\hline
		irrep       & Basics funtioncs                           & Band \\ [0.6ex]
		\hline
		$A_{1}^{'}$ & $\ket{\Psi_{2,0}}$                         & VB   \\
		$E_{}^{'}$  & $\{ \ket{\Psi_{2,2}},\ket{\Psi_{2,-2}} \}$ & VB-3 \\
		$E_{}^{''}$ & $\{ \ket{\Psi_{2,1}},\ket{\Psi_{2,-1}} \}$ & VB-1 \\
		\hline
	\end{tabular}
	\caption{Cơ sở cho cho biểu diển bất khả quy của nhóm $D_{3h}$ tại điểm $\Gamma$}.
\end{table}










\end{document}

