\documentclass{report}
\usepackage[utf8]{vietnam}
\usepackage[utf8]{inputenc}
\usepackage{anyfontsize,fontsize}
\changefontsize[13pt]{13pt}
\usepackage{commath}
\usepackage{blindtext}
\usepackage{parskip}
\usepackage{xcolor}
\usepackage{amssymb}
\usepackage{slashed}
\usepackage{indentfirst}
\usepackage{pdfpages}
\usepackage{graphicx}
%\usepackage{tikz-feynman}
\usepackage{nccmath}
\usepackage{mathtools}
\usepackage{amsfonts}
\usepackage{amsmath,systeme,bbold}
\usepackage[thinc]{esdiff}
\usepackage{hyperref}
\usepackage{dirtytalk,bm,physics}
\usepackage{tikz}
\usepackage{lipsum}
\usepackage{fancyhdr}
%footnote
\pagestyle{fancy}
\renewcommand{\headrulewidth}{0pt}%
\fancyhf{}%
\fancyfoot[L]{Vật lý Lý thuyết}%
\fancyfoot[C]{\hspace{4cm} \thepage}%

\usetikzlibrary{shapes.geometric, arrows}

\usepackage{geometry}
\geometry{
    a4paper,
    total={170mm,257mm},
    left=20mm,
    top=20mm,
}

\renewcommand{\baselinestretch}{2.0}
\usetikzlibrary{arrows.spaced}
\usetikzlibrary{animations,quotes}
%gian do
\tikzstyle{startstop} = [rectangle, rounded corners, minimum width=3cm, minimum height=1cm, text centered,draw=black, fill=white!30]
\tikzstyle{arrow} = [thick,->,>=stealth]

\title{\Huge{ }}

\hypersetup{
    colorlinks=true,
    linkcolor=red,
    filecolor=magenta,
    urlcolor=cyan,
    pdftitle={},
    pdfpagemode=FullScreen,
}

\urlstyle{same}

\begin{document}
\setlength{\parindent}{20pt}
\newpage
\author{TRẦN KHÔI NGUYÊN \\ VẬT LÝ LÝ THUYẾT}
\maketitle




Từ Hamiltonian $H^{jj'}_{\mu\mu'} \left( \mathbf{k}\right) =  \sum_{\mathbf{R}} e^{i \mathbf{k\cdot R}} E^{jj'}_{\mu\mu'} \left(\textbf{R}\right) $
trong đó
\begin{align*}
    E^{jj'}_{\mu\mu'} \left(\textbf{R}\right) = \bra*{\phi^j_\mu \left(\textbf{r}\right)}\hat{H}\ket*{\phi^{j'}_{\mu'}\left(\textbf{r} - \textbf{R}\right)}
\end{align*}

\begin{align*}
    \ket*{\phi^1_1} = d_{z^2} , \quad \ket*{\phi^2_1} = d_{xy} , \quad \ket*{\phi^2_2} = d_{x^2 - y^2}
\end{align*}

\begin{align}
    H^{jj'}_{\mu\mu'} \left( \mathbf{k}\right) & =  \; \sum_{\mu\mu'}^{jj'}
    e^{i \mathbf{k\cdot R_1}} E^{jj'}_{\mu\mu'} \left(\mathbf{R_1}\right)
    + \sum_{\mu\mu'}^{jj'} e^{i \mathbf{k\cdot R_2}} E^{jj'}_{\mu\mu'} \left(\mathbf{R_2}\right)
    + \sum_{\mu\mu'}^{jj'} e^{i \mathbf{k\cdot R_3}} E^{jj'}_{\mu\mu'} \left(\mathbf{R_3}\right) \nonumber                                    \\
                                               & + \sum_{\mu\mu'}^{jj'} e^{i \mathbf{k\cdot R_4}} E^{jj'}_{\mu\mu'} \left(\mathbf{R_4}\right)
    + \sum_{\mu\mu'}^{jj'} e^{i \mathbf{k\cdot R_5}} E^{jj'}_{\mu\mu'} \left(\mathbf{R_5}\right)
    + \sum_{\mu\mu'}^{jj'} e^{i \mathbf{k\cdot R_6}} E^{jj'}_{\mu\mu'} \left(\mathbf{R_6}\right) \nonumber
\end{align}

\[
    \renewcommand{\arraystretch}{0.75}
    H^{NN} = \begin{bmatrix}
        h_{0}   & h_{1}    & h_{2}  \\
        h_{1}^* & h_{11}   & h_{12} \\
        h_{2}^* & h_{12}^* & h_{22}
    \end{bmatrix}
\]

\begin{align}
    h_0 = \sum_{R} e^{i\mathbf{k\cdot R}} \bra*{\phi_1^1 \left(\mathbf{r}\right)} H \ket*{\phi_1^1\left(\mathbf{r}-\mathbf{R}\right)}; \quad h_1 = \sum_{R} e^{i\mathbf{k\cdot R}} \bra*{\phi_1^1 \left(\mathbf{r}\right)} H \ket*{\phi_1^2\left(\mathbf{r}-\mathbf{R}\right)} \nonumber     \\
    h_2 = \sum_{R} e^{i\mathbf{k\cdot R}} \bra*{\phi_1^1 \left(\mathbf{r}\right)} H \ket*{\phi_2^2\left(\mathbf{r}-\mathbf{R}\right)}; \quad  h_{11} = \sum_{R} e^{i\mathbf{k\cdot R}} \bra*{\phi_1^2 \left(\mathbf{r}\right)} H \ket*{\phi_1^2\left(\mathbf{r}-\mathbf{R}\right)} \nonumber \\
    h_{12} = \sum_{R} e^{i\mathbf{k\cdot R}} \bra*{\phi_1^2 \left(\mathbf{r}\right)} H \ket*{\phi_2^2\left(\mathbf{r}-\mathbf{R}\right)}; \quad h_{22} = \sum_{R} e^{i\mathbf{k\cdot R}} \bra*{\phi_2^2 \left(\mathbf{r}\right)} H \ket*{\phi_2^2\left(\mathbf{r}-\mathbf{R}\right)} \nonumber
\end{align}



Lại có $E^{jj'} \left(\hat{g_n}\textbf{R}\right)= D^j(\hat{g_n}) E^{jj'} \left(\textbf{R}\right)\left[D^j(\hat{g_n})\right]^\dagger$


trong đó $\hat{g_n} = \{E,C_3,C^2_3,\sigma_\nu,\sigma'_\nu,\sigma''_\nu\}$


trong đó $D^1(\hat{g_n}) = 1$
\[
    \renewcommand{\arraystretch}{0.75}
    D^2(E) = \begin{bmatrix}
        1 & 0 \\
        0 & 1
    \end{bmatrix}
\]
\[
    \renewcommand{\arraystretch}{0.75}
    D^2(\hat{C_3}) = \begin{bmatrix}
        cos\phi & -sin\phi \\
        sin\phi & cos\phi
    \end{bmatrix}
    = \begin{bmatrix}
        -\frac{1}{2}       & -\frac{\sqrt{3}}{2} \\
        \frac{\sqrt{3}}{2} & -\frac{1}{2}
    \end{bmatrix}
\]
\[
    \renewcommand{\arraystretch}{0.75}
    D^2(\hat{C_3}^2) =
    \begin{bmatrix}
        -\frac{1}{2}        & \frac{\sqrt{3}}{2} \\
        \frac{-\sqrt{3}}{2} & -\frac{1}{2}
    \end{bmatrix}
\]

Để tìm được $D^2(\sigma_\nu)$ ta cố định $\bigtriangleup$ ABC : A($\dfrac{1}{2},\dfrac{\sqrt{3}}{2}$),B(1,0), C(0,0).

Khi đổi chỗ A $\leftrightarrow$ B, ta được ma trận:

\[
    \renewcommand{\arraystretch}{0.75}
    \begin{bmatrix}
        1 \\
        0
    \end{bmatrix}
    = D^2(\sigma_\nu)
    \begin{bmatrix}
        \frac{1}{2} \\
        \frac{\sqrt{3}}{2}
    \end{bmatrix}
    \Rightarrow  D^2(\sigma_\nu) =
    \begin{bmatrix}
        \frac{1}{2}        & \frac{\sqrt{3}}{2} \\
        \frac{\sqrt{3}}{2} & -\frac{1}{2}
    \end{bmatrix}
\]

trong đó $ D^2\left(\sigma_\nu'\right) = D^2 \left(C_3\right) D^2\left(\sigma_\nu\right)\quad ; D^2\left(\sigma_\nu''\right) = D^2 \left(C_3^2\right) D^2\left(\sigma_\nu\right) $


\[
    \renewcommand{\arraystretch}{0.75}
    D^2\left(\sigma'_\nu\right) =
    \begin{bmatrix}
        -1 & 0 \\
        0  & 1
    \end{bmatrix}
\]
\[
    \renewcommand{\arraystretch}{0.75}
    D^2\left(\sigma''_\nu\right) =
    \begin{bmatrix}
        \frac{1}{2}  & \frac{\sqrt{3}}{2} \\
        -\frac{3}{2} & -\frac{1}{2}
    \end{bmatrix}
\]

Toán tử $C_3$ đánh lên $\textbf{R}_1$ ta được $\rightarrow$ $\textbf{R}_5$ (dưới dạng ma trận)

Toán tử $C_3^2$ đánh lên $\textbf{R}_1$ ta được $\rightarrow$ $\textbf{R}_3$ (dưới dạng ma trận)

Toán tử $\sigma_\nu$ đánh lên $\textbf{R}_1$ ta được $\rightarrow$ $\textbf{R}_6$ (dưới dạng ma trận)

Toán tử $\sigma'_\nu$ đánh lên $\textbf{R}_1$ ta được $\rightarrow$ $\textbf{R}_4$ (dưới dạng ma trận)

Toán tử $\sigma''_\nu$ đánh lên $\textbf{R}_1$ ta được $\rightarrow$ $\textbf{R}_2$ (dưới dạng ma trận)

Kiểm tra điều trên:

\[
    \renewcommand{\arraystretch}{0.75}
    D^2\left(C_3^2\right)R_1 =
    \begin{bmatrix}
        -\frac{1}{2} & \frac{\sqrt{3}}{2} \\
        -\frac{3}{2} & -\frac{1}{2}
    \end{bmatrix}
    \begin{bmatrix}
        1 \\
        0
    \end{bmatrix}
    =
    \begin{bmatrix}
        -\frac{1}{2} \\
        -\frac{\sqrt{3}}{2}
    \end{bmatrix}
    = \textbf{R}_3
\]
\[
    \renewcommand{\arraystretch}{0.75}
    D^2\left(\sigma''_\nu\right)R_1 =
    \begin{bmatrix}
        \frac{1}{2}  & -\frac{\sqrt{3}}{2} \\
        -\frac{3}{2} & -\frac{1}{2}
    \end{bmatrix}
    \begin{bmatrix}
        1 \\
        0
    \end{bmatrix}
    =
    \begin{bmatrix}
        \frac{1}{2} \\
        -\frac{\sqrt{3}}{2}
    \end{bmatrix}
    = \textbf{R}_2
\]


\textbf{ $\ast$ h0}

\begin{align}
    h_0 & = \sum_{\mathbf{R} \neq 0} e^{i\mathbf{k\cdot R}} \bra*{\phi_1^1 \left(\mathbf{r}\right)} H \ket*{\phi_1^1\left(\mathbf{r}-\mathbf{R}\right)}
    + \bra*{\phi_1^1 \left(\mathbf{r}\right)} H \ket*{\phi_1^1\left(\mathbf{r}\right)}  \nonumber                                                                                                                                                                                                                                         \\
        & = e^{i\mathbf{k\cdot R_1}} \bra*{\phi_1^1 \left(\mathbf{r}\right)} H \ket*{\phi_1^1\left(\mathbf{r}-\mathbf{R_1}\right)} + e^{i\mathbf{k\cdot R_4}} \bra*{\phi_1^1 \left(\mathbf{r}\right)} H \ket*{\phi_1^1\left(\mathbf{r}-\mathbf{R_4}\right)} \nonumber                                                                     \\
        & + e^{i\mathbf{k\cdot R_2}} \bra*{\phi_1^1 \left(\mathbf{r}\right)} H \ket*{\phi_1^1\left(\mathbf{r}-\mathbf{R_2}\right)} + e^{i\mathbf{k\cdot R_5}} \bra*{\phi_1^1 \left(\mathbf{r}\right)} H \ket*{\phi_1^1\left(\mathbf{r}-\mathbf{R_5}\right)} \nonumber                                                                     \\
        & + e^{i\mathbf{k\cdot R_3}} \bra*{\phi_1^1 \left(\mathbf{r}\right)} H \ket*{\phi_1^1\left(\mathbf{r}-\mathbf{R_3}\right)} + e^{i\mathbf{k\cdot R_6}} \bra*{\phi_1^1 \left(\mathbf{r}\right)} H \ket*{\phi_1^1\left(\mathbf{r}-\mathbf{R_6}\right)} + \epsilon_1 \nonumber                                                        \\
        & = e^{i k_x a} E_{11}^{11}\left(\mathbf{R_1}\right) + e^{-i k_x a} E_{11}^{11}\left(\mathbf{R_4}\right) +  e^{i \left( k_x \frac{a}{2} - k_y \frac{a\sqrt{3}}{2} \right)} E_{11}^{11}\left(\mathbf{R_2}\right) +  e^{-i \left( k_x \frac{a}{2} - k_y \frac{a\sqrt{3}}{2} \right)} E_{11}^{11}\left(\mathbf{R_5}\right) \nonumber \\
        & + e^{-i \left( k_x \frac{a}{2} + k_y \frac{a\sqrt{3}}{2} \right)} E_{11}^{11}\left(\mathbf{R_3}\right) + e^{i \left( k_x \frac{a}{2} + k_y \frac{a\sqrt{3}}{2} \right)} E_{11}^{11}\left(\mathbf{R_6}\right) + \epsilon_1 \nonumber                                                                                             \\
        & = 2 E_{11}^{11}\left(\mathbf{R_1}\right) \left(cos2\alpha + 2cos\alpha\; cos\beta\right) + \epsilon_1  \nonumber
\end{align}
\clearpage

\textbf{ $\ast$ h1}
\begin{align}
    h_1 & = \sum_{\mathbf{R} \neq 0} e^{i\mathbf{k\cdot R}} \bra*{\phi_1^1 \left(\mathbf{r}\right)} H \ket*{\phi_1^2\left(\mathbf{r}-\mathbf{R}\right)}\nonumber                                                                                                                                                                        \\
        & = e^{i k_x a} E_{11}^{12}\left(\mathbf{R_1}\right) + e^{-i k_x a} E_{11}^{12}\left(\mathbf{R_4}\right) + e^{i \left( k_x \frac{a}{2} - k_y \frac{a\sqrt{3}}{2} \right)} E_{11}^{12}\left(\mathbf{R_2}\right) + e^{-i \left( k_x \frac{a}{2} - k_y \frac{a\sqrt{3}}{2} \right)} E_{11}^{12}\left(\mathbf{R_5}\right) \nonumber \\
        & + e^{-i \left( k_x \frac{a}{2} + k_y \frac{a\sqrt{3}}{2} \right)} E_{11}^{12}\left(\mathbf{R_3}\right) + e^{i \left( k_x \frac{a}{2} + k_y \frac{a\sqrt{3}}{2} \right)} E_{11}^{12}\left(\mathbf{R_6}\right) \nonumber
\end{align}

trong đó
\begin{align}
    E^{12}(\mathbf{R_2})
     & = E^{12}(\sigma''_\nu \mathbf{R_1}) = D^1(\sigma''_\nu) E^{12}(\mathbf{R_1}) \left[ D^2(\sigma''_\nu)\right]^\dagger \nonumber                                                                                                                                                                                    \\
     & = \begin{bmatrix}
             1
         \end{bmatrix}
    \begin{bmatrix}
        E_{11}^{12}(\mathbf{R_1}) & E_{12}^{12}(\mathbf{R_1})
    \end{bmatrix}
    \begin{bmatrix}
        \frac{1}{2}         & -\frac{\sqrt{3}}{2} \\
        -\frac{\sqrt{3}}{2} & -\frac{1}{2}
    \end{bmatrix} \nonumber                                                                                                                                                                                                                                                             \\
     & = \begin{bmatrix}
             \dfrac{ E_{11}^{12}(\mathbf{R_1}) - \sqrt{3} E_{12}^{12}(\mathbf{R_1})}{2} & \dfrac{ - E_{11}^{12}(\mathbf{R_1})\sqrt{3} -  E_{12}^{12}(\mathbf{R_1})}{2}
         \end{bmatrix}
    \nonumber
\end{align}
$\Rightarrow E_{11}^{12}(\mathbf{R_2}) =  \dfrac{ E_{11}^{12}(\mathbf{R_1}) - \sqrt{3} E_{12}^{12}(\mathbf{R_1})}{2} $

Tương tự ta có cho:

\begin{align}
    E_{11}^{12}(\mathbf{R_3}) & =  \dfrac{ -E_{11}^{12}(\mathbf{R_1}) + \sqrt{3} E_{12}^{12}(\mathbf{R_1})}{2} \quad ;  E_{11}^{12}(\mathbf{R_4}) = - E_{11}^{12}(\mathbf{R_1}) \nonumber                                                 \\
    E_{11}^{12}(\mathbf{R_5}) & =  \dfrac{ -E_{11}^{12}(\mathbf{R_1}) - \sqrt{3} E_{12}^{12}(\mathbf{R_1})}{2} \quad ; E_{11}^{12}(\mathbf{R_6})  =  \dfrac{ E_{11}^{12}(\mathbf{R_1}) + \sqrt{3} E_{12}^{12}(\mathbf{R_1})}{2} \nonumber
\end{align}

\begin{align}
    h_1 & = e^{i2\alpha} E_{11}^{12}(\mathbf{R_1}) - e^{i2\alpha} E_{11}^{12}(\mathbf{R_1})   \nonumber                                                                                                                   \\
        & + e^{i(\alpha - \beta)}\dfrac{ E_{11}^{12}(\mathbf{R_1}) - \sqrt{3} E_{12}^{12}(\mathbf{R_1})}{2} + e^{-i(\alpha + \beta)}\dfrac{ -E_{11}^{12}(\mathbf{R_1}) + \sqrt{3} E_{12}^{12}(\mathbf{R_1})}{2} \nonumber \\
        & + e^{i(-\alpha + \beta)} \dfrac{ -E_{11}^{12}(\mathbf{R_1}) - \sqrt{3} E_{12}^{12}(\mathbf{R_1})}{2} + e^{i(\alpha+\beta)}\dfrac{ E_{11}^{12}(\mathbf{R_1}) + \sqrt{3} E_{12}^{12}(\mathbf{R_1})}{2} \nonumber  \\
        & = 2isin2\alpha E_{11}^{12}(\mathbf{R}_1) + 2i \dfrac{E_{11}^{12}(\mathbf{R_1})}{2} sin(\alpha-\beta) - 2\dfrac{E_{12}^{12}(\mathbf{R_1}\sqrt{3})}{2} cos(\alpha - \beta) \nonumber                              \\
        & +  2i \dfrac{E_{11}^{12}(\mathbf{R_1})}{2} sin(\alpha+\beta) + 2\dfrac{E_{12}^{12}(\mathbf{R_1}\sqrt{3})}{2} cos(\alpha - \beta) \nonumber                                                                      \\
        & = -2\sqrt{3} t_2 sin\alpha \; sin\beta + 2i t_1(sin2\alpha + sin\alpha\;\cos\beta) \nonumber
\end{align}


\clearpage

Đặt
\begin{align}
    t_0 = E_{11}^{11}(\mathbf{R}_1); \quad
    t_1 = E_{11}^{12}(\mathbf{R}_1); \quad
    t_2 = E_{12}^{12}(\mathbf{R}_1); \quad \nonumber \\
    t_{11} = E_{11}^{22}(\mathbf{R}_1); \quad
    t_{12} = E_{12}^{22}(\mathbf{R}_1); \quad
    c_{21} = E_{21}^{22}(\mathbf{R}_1); \quad
    t_{22} = E_{22}^{22}(\mathbf{R}_1); \quad \nonumber
\end{align}

\textbf{ $\ast$ h22}

\begin{align}
    h_{22} & = \sum_{R}^{} e^{i\mathbf{k\cdot R}} E^{22}_{22}(\textbf{R}) \nonumber                                                                                                                                \\
           & =  e^{i\mathbf{k\cdot R_1}} E^{22}_{22}(\mathbf{R_1}) +  e^{i\mathbf{k\cdot R_2}} E^{22}_{22}(\mathbf{R_2}) +  e^{i\mathbf{k\cdot R_3}} E^{22}_{22}(\mathbf{R_3}) \nonumber                           \\
           & +  e^{i\mathbf{k\cdot R_4}} E^{22}_{22}(\mathbf{R_4}) +  e^{i\mathbf{k\cdot R_5}} E^{22}_{22}(\mathbf{R_5}) +  e^{i\mathbf{k\cdot R_6}} E^{22}_{22}(\mathbf{R_6}) + E^{22}_{22}(\mathbf{0}) \nonumber
\end{align}

\begin{align}
    E^{22} (\mathbf{R}_2)
     & =  E^{22} (\sigma''_\nu \mathbf{R}_1) \nonumber                                              \\
     & = D^2(\sigma''_\nu)  E^{22}(\mathbf{R}_1) \left[ D^2(\sigma''_\nu) \right]^\dagger \nonumber
\end{align}

\[
    \renewcommand{\arraystretch}{0.75}
    =
    \begin{bmatrix}
        \frac{1}{2}  & -\frac{\sqrt{3}}{2} \\
        -\frac{3}{2} & -\frac{1}{2}
    \end{bmatrix}
    \begin{bmatrix}
        E_{11}^{22}(\mathbf{R}_1) & E_{12}^{22}(\mathbf{R}_1) \\
        E_{21}^{22}(\mathbf{R}_1) & E_{22}^{22}(\mathbf{R}_1)
    \end{bmatrix}
    \begin{bmatrix}
        \frac{1}{2}  & -\frac{\sqrt{3}}{2} \\
        -\frac{3}{2} & -\frac{1}{2}
    \end{bmatrix}
\]

\[
    \renewcommand{\arraystretch}{0.75}
    =
    \begin{bmatrix}
        \frac{t_{11} - t_{12}\sqrt{3} -c_{21}\sqrt{3} + 3t_{22}}{4}   & \frac{-t_{11}\sqrt{3} - t_{12} + 3c_{21} + \sqrt{3}t_{22}}{4} \\
        \frac{-t_{11}\sqrt{3} + 3t_{12} - c_{21} + \sqrt{3}t_{22}}{4} & \frac{3t_{11} + t_{12}\sqrt{3} + c\sqrt{3} + t_{22}}{4}
    \end{bmatrix}
\]

$\Rightarrow E_{22}^{22}(\mathbf{R}_2) = \dfrac{3t_{11} + t_{12}\sqrt{3} + c\sqrt{3} + t_{22}}{4}$

Tương tự ta có cho:

\begin{align}
    E_{22}^{22}(\mathbf{R}_3) & = \dfrac{3t_{11} - t_{12}\sqrt{3} - c\sqrt{3} + t_{22}}{4} \nonumber \\
    E_{22}^{22}(\mathbf{R}_4) & = t_{22} \nonumber                                                   \\
    E_{22}^{22}(\mathbf{R}_5) & = \dfrac{3t_{11} + t_{12}\sqrt{3} + c\sqrt{3} + t_{22}}{4} \nonumber \\
    E_{22}^{22}(\mathbf{R}_6) & = \dfrac{3t_{11} - t_{12}\sqrt{3} - c\sqrt{3} + t_{22}}{4} \nonumber
\end{align}
\clearpage
Ta được:

\begin{align}
    h_{22} & = e^{ i 2 \alpha} t_{22} + e^{- i 2 \alpha} t_{22} \nonumber                                                                                                                                                                           \\
           & + e^{i(\alpha - \beta)} \left(\dfrac{3t_{11} + t_{12}\sqrt{3} + c\sqrt{3} + t_{22}}{4}\right) +  e^{-i(\alpha + \beta)} \left(\dfrac{3t_{11} - t_{12}\sqrt{3} - c\sqrt{3} + t_{22}}{4}\right) \nonumber                                \\
           & + e^{i(-\alpha + \beta)} \left(\dfrac{3t_{11} + t_{12}\sqrt{3} + c\sqrt{3} + t_{22}}{4}\right) +  e^{i(\alpha + \beta)} \left(\dfrac{3t_{11} - t_{12}\sqrt{3} - c\sqrt{3} + t_{22}}{4}\right) \nonumber                                \\
           & = 2 cos(2\alpha) t_{22} + \dfrac{1}{4}3t_{11} \left(e^{i\alpha} + e^{-i\alpha}\right)\left(e^{-i\beta} + e^{i\beta}\right) + \dfrac{1}{4}t_{22} \left(e^{i\alpha} + e^{-i\alpha}\right)\left(e^{-i\beta} + e^{i\beta}\right) \nonumber \\
           & + c\sqrt{3}(e^{i(\alpha-\beta)} - e^{i(-\alpha+\beta)} +e^{i(-\alpha+\beta)} - e^{i(\alpha+\beta)}) \nonumber                                                                                                                          \\
           & + t_{12}\sqrt{3}(e^{i(\alpha-\beta)} - e^{i(-\alpha+\beta)} + e^{i(-\alpha+\beta)} - e^{i(\alpha+\beta)}) \nonumber                                                                                                                    \\
           & = 2cos(2\alpha) t_{22} + (3t_{11} + t_{22})cos\alpha \, cos\beta \nonumber
\end{align}
Sử dụng tính Hermite của Hamiltonian $h_{22}$ là số thực, nên $t_{12}= - t_{21}$

\textbf{**h11}
\begin{align*}
    H^{22}_{11} & = \sum_{\mathbf{R}} e^{i \mathbf{k} \cdot \mathbf{R}} E^{22}_{11}(\mathbf{R})
    = e^{i \mathbf{k} \cdot \mathbf{R}_1} E^{22}_{11}(\mathbf{R}_1)
    + e^{i \mathbf{k} \cdot \mathbf{R}_2} E^{22}_{11}(\mathbf{R}_2)
    + e^{i \mathbf{k} \cdot \mathbf{R}_3} E^{22}_{11}(\mathbf{R}_3)                                                                                                                                                                 \\
                & + e^{i \mathbf{k} \cdot \mathbf{R}_4} E^{22}_{11}(\mathbf{R}_4)
    + e^{i \mathbf{k} \cdot \mathbf{R}_5} E^{22}_{11}(\mathbf{R}_5)
    + e^{i \mathbf{k} \cdot \mathbf{R}_6} E^{22}_{11}(\mathbf{R}_6)
    + E^{22}_{11}(\mathbf{0})                                                                                                                                                                                                       \\
                & = e^{i k_x a} E^{22}_{11}(\mathbf{R}_1)
    + e^{i \left( k_x \frac{a}{2} - k_y \frac{a\sqrt{3}}{2} \right)} E^{22}_{11}(\mathbf{R}_2)
    + e^{i \left( -k_x \frac{a}{2} - k_y \frac{a\sqrt{3}}{2} \right)} E^{22}_{11}(\mathbf{R}_3)                                                                                                                                     \\
                & + e^{-i k_x a } E^{22}_{11}(\mathbf{R}_4)
    + e^{i \left( -k_x \frac{a}{2} + k_y \frac{a\sqrt{3}}{2} \right)} E^{22}_{11}(\mathbf{R}_5)
    + e^{i \left( k_x \frac{a}{2} + k_y \frac{a\sqrt{3}}{2} \right)} E^{22}_{11}(\mathbf{R}_6) + \epsilon_2                                                                                                                         \\
                & = e^{2i\alpha} t_{11} + e^{i \left( \alpha - \beta \right)} \frac{t_{11} -\sqrt{3}t_{12} - \sqrt{3}c_{21} + 3t_{22}}{4}                                                                                           \\
                & + e^{i \left( -\alpha - \beta \right)} \frac{t_{11} -\sqrt{3}t_{12} - \sqrt{3}c_{21} + 3t_{22}}{4} + e^{-2i\alpha} t_{11}                                                                                         \\
                & + e^{i \left( -\alpha + \beta \right)} \frac{t_{11} +\sqrt{3}t_{12} + \sqrt{3}c_{21} + 3t_{22}}{4} + e^{i \left( \alpha + \beta \right)} \frac{t_{11} +\sqrt{3}t_{12} + \sqrt{3}c_{21} + 3t_{22}}{4} + \epsilon_2 \\
                & = 2t_{11}cos(2\alpha) + \left (t_{11} + 3t_{22}\right )cos(\alpha)cos(\beta) + \epsilon_2                                                                                                                         \\
                & \textbf{Lưu ý ở đây đã sử dụng tính chất Hermite của $h_{11}$ phải là số thực}                                                                                                                                    \\ &\Rightarrow t_{12} = - t_{21}
\end{align*}

\begin{align*}
    E^{22} (\mathbf{R_2}) & = E^{22}(\sigma_\nu^{''} \mathbf{R_1}) = D^2(\sigma^{''}_\nu) E^{22} (\mathbf{R_1}) [D^2(\sigma_\nu ^{''})]^\dagger \\
                          & =
    \begin{bmatrix}
        \frac{1}{2}         & -\frac{\sqrt{3}}{2} \\
        -\frac{\sqrt{3}}{2} & -\frac{1}{2}
    \end{bmatrix} \begin{bmatrix}
                      a & b \\
                      c & d
                  \end{bmatrix}\begin{bmatrix}
                                   \frac{1}{2}         & -\frac{\sqrt{3}}{2} \\
                                   -\frac{\sqrt{3}}{2} & -\frac{1}{2}
                               \end{bmatrix}\text{Trong đó}\begin{bmatrix}
                                                               a = t_{11} \\
                                                               b = t_{12} \\
                                                               c = c_{21} \\
                                                               d = t_{22}
                                                           \end{bmatrix}                                                         \\
                          & \Rightarrow E^{22}_{11} (\mathbf{R_2}) = \frac{a - \sqrt{3}b - \sqrt{3}c + 3d}{4}
\end{align*}
Tương tự ta tìm được:
\begin{align*}
    E^{22}_{11} (\mathbf{R_3}) & = \frac{a - \sqrt{3}b - \sqrt{3}c + 3d}{4} \\
    E^{22}_{11} (\mathbf{R_4}) & = a                                        \\
    E^{22}_{11} (\mathbf{R_5}) & = \frac{a + \sqrt{3}b + \sqrt{3}c + 3d}{4} \\
    E^{22}_{11} (\mathbf{R_6}) & = \frac{a + \sqrt{3}b + \sqrt{3}c + 3d}{4} \\
\end{align*}

\textbf{**h12}
\begin{align*}
    H^{22}_{12} & = \sum_{\mathbf{R}} e^{i \mathbf{k} \cdot \mathbf{R}} E^{22}_{12}(\mathbf{R})                             \\
                & = e^{i \mathbf{k} \cdot \mathbf{R}_1} E^{22}_{12}(\mathbf{R}_1)
    + e^{i \mathbf{k} \cdot \mathbf{R}_2} E^{22}_{12}(\mathbf{R}_2)
    + e^{i \mathbf{k} \cdot \mathbf{R}_3} E^{22}_{12}(\mathbf{R}_3)                                                         \\
                & \quad + e^{i \mathbf{k} \cdot \mathbf{R}_4} E^{22}_{12}(\mathbf{R}_4)
    + e^{i \mathbf{k} \cdot \mathbf{R}_5} E^{22}_{12}(\mathbf{R}_5)
    + e^{i \mathbf{k} \cdot \mathbf{R}_6} E^{22}_{12}(\mathbf{R}_6)
    + E^{22}_{12}(\mathbf{0})                                                                                               \\
                & = e^{i k_x a} E^{22}_{12}(\mathbf{R}_1)
    + e^{i \left( k_x \frac{a}{2} - k_y \frac{a\sqrt{3}}{2} \right)} E^{22}_{12}(\mathbf{R}_2)                              \\
                & \quad + e^{i \left( -k_x \frac{a}{2} - k_y \frac{a\sqrt{3}}{2} \right)} E^{22}_{12}(\mathbf{R}_3)         \\
                & \quad + e^{-i k_x a } E^{22}_{12}(\mathbf{R}_4)
    + e^{i \left( -k_x \frac{a}{2} + k_y \frac{a\sqrt{3}}{2} \right)} E^{22}_{12}(\mathbf{R}_5)                             \\
                & \quad + e^{i \left( k_x \frac{a}{2} + k_y \frac{a\sqrt{3}}{2} \right)} E^{22}_{12}(\mathbf{R}_6)          \\
                & = e^{2i\alpha} t_{12}
    + e^{i \left( \alpha - \beta \right)} \frac{-\sqrt{3}t_{11} - t_{12} + 3c_{21} + \sqrt{3}t_{22}}{4}                     \\
                & \quad + e^{i \left( -\alpha - \beta \right)} \frac{\sqrt{3}t_{11} + t_{12} - 3c_{21} - \sqrt{3}t_{22}}{4} \\
                & \quad - e^{-2i\alpha} t_{12}
    + e^{i \left( -\alpha + \beta \right)} \frac{-\sqrt{3}t_{11} + t_{12} - 3c_{21} + \sqrt{3}t_{22}}{4}                    \\
                & \quad + e^{i \left( \alpha + \beta \right)} \frac{\sqrt{3}t_{11} - t_{12} + 3c_{21} - \sqrt{3}t_{22}}{4}  \\
                & = \sqrt{3}(t_{22} - t_{11}) \sin(\alpha) \sin(\beta)
    + 4i t_{12} \sin(\alpha) (\cos(\alpha) - \cos(\beta)) + 3i c_{21} sin(\alpha)cos(\beta)
\end{align*}

\begin{align*}
    E^{22} (\mathbf{R_2}) & = E^{22}(\sigma_\nu^{''} \mathbf{R_1}) = D^2(\sigma^{''}_\nu) E^{22} (\mathbf{R_1}) [D^2(\sigma_\nu ^{''})]^\dagger \\
                          & =
    \begin{bmatrix}
        \frac{1}{2}         & -\frac{\sqrt{3}}{2} \\
        -\frac{\sqrt{3}}{2} & -\frac{1}{2}
    \end{bmatrix} \begin{bmatrix}
                      a & b \\
                      c & d
                  \end{bmatrix}\begin{bmatrix}
                                   \frac{1}{2}         & -\frac{\sqrt{3}}{2} \\
                                   -\frac{\sqrt{3}}{2} & -\frac{1}{2}
                               \end{bmatrix}\text{Trong đó}\begin{bmatrix}
                                                               a = t_{11} \\
                                                               b = t_{12} \\
                                                               c = c_{21} \\
                                                               d = t_{22}
                                                           \end{bmatrix}                                                         \\
                          & \Rightarrow E^{22}_{12} (\mathbf{R_2}) = \frac{-\sqrt{3}a - b + 3c + \sqrt{3}d}{4}
\end{align*}
Tương tự ta tìm được:
\begin{align*}
    E^{22}_{12} (\mathbf{R_3}) & = \frac{\sqrt{3}a + b - 3c - \sqrt{3}d}{4} \\
    E^{22}_{12} (\mathbf{R_4}) & = -b                                       \\
    E^{22}_{12} (\mathbf{R_5}) & = \frac{\sqrt{3}a + b - 3c + \sqrt{3}d}{4} \\
    E^{22}_{12} (\mathbf{R_6}) & = \frac{\sqrt{3}a - b + 3c - \sqrt{3}d}{4} \\
\end{align*}

\end{document}

