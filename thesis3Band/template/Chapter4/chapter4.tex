\chapter{Thuật toán, mã nguồn}
\label{Chapter4}

\section{Thuật toán, mã giả}
Để soạn thảo thuật toán hoặc mã giả trong LaTeX, bạn có thể sử dụng package \textit{algorithm}:
\begin{lstlisting}
\usepackage{algorithm}
\end{lstlisting}

Có thể chèn thuật toán/mã giả như sau:

\begin{algorithm}
\caption{An algorithm with caption}\label{alg:cap}
\begin{algorithmic}[1]
\Require $n \geq 0$
\Ensure $y = x^n$
\State $y \gets 1$
\State $X \gets x$
\State $N \gets n$
\While{$N \neq 0$}
\If{$N$ is even}
    \State $X \gets X \times X$
    \State $N \gets \frac{N}{2}$  \Comment{This is a comment}
\ElsIf{$N$ is odd}
    \State $y \gets y \times X$
    \State $N \gets N - 1$
\EndIf
\EndWhile
\end{algorithmic}
\end{algorithm}

Lưu ý rằng tên lệnh do 'algpseudocode' cung cấp thường được viết hoa chữ cái đầu, ví dụ: $\backslash${\tt{State}}, $\backslash${\tt{While}}, $\backslash${\tt{EndWhile}}.


\section{Mã nguồn}
Để chèn mã nguồn, cần dùng package \textit{listings}:

\begin{lstlisting}
\usepackage{listings}
\end{lstlisting}

Mã nguồn có thể được chèn trực tiếp như sau:

\begin{lstlisting}[language=C++]
cout << "Hello, World!" << endl;
\end{lstlisting}
hoặc chèn thông qua tập tin chứa mã nguồn trong thư mục $SourceCode$ như sau:

\lstinputlisting[language=Python]{SourceCode/hello.py}





