
\section{Conclusions and Outlook}
In our research, we have calculated the Hofstadter butterfly of monolayer MoS$_{2}$ and others transistion metal dichalcogenide types by using a tight-binding three-band model. In addition, we have explored the rich and complex physics of monolayer MoS$_2$, such as Landau levels and integer quantum Hall effect (IQHE), in the presence of external magnetic fields. The research conducted within these pages has demonstated the unique interplay between the superlattice and magnetic fields, which leads to the emergence of fascinating quantum phenomena.

In section 2.1, we have studied the tight-binding three-band model for monolayers of MX$_{2}$ using only the M$-d_{z^{2}}$, $d_{xy}$ and $d_{x^{2} - y{^2}}$ orbitals. When TNN M-M hoppings are included, we calculated the hopping energies using the symmetry of the $D_{3h}$ point group we derived nineteen hopping parameters from Ref \cite{PhysRevB.88.085433}.

In section 2.2, we focused on the Hofstadter physics in monolayer TMD, where the lattice gives rise to a rich Hofstadter spectrum when subjected to a magnetic field. The detailed analysis revealed key features of the spectrum, including the SOC and the emergence of topological quantum Hall states. 
%In addition, the study also demonstrated that there are many ways to derivive the Hofstadter spectrum two of those is using the Peierls substitution or Envelope Function Approximation.

In section 2.3 and section 2.4, extended the investigation into the realm of Hall effects, introducing the Landau levels, the integer quantum Hall effect and applying it to monolayer TMD systems. We also shown that how the Hofstadter butterfly can be colored in various ways by using the Chern number.

Overall, while this study provides valuable insights, we acknowledge several limitations. Firstly, in section 2.3, our calculation was restrited to the single-band approximation due to the computational complexity of multi-band interactions. Specifically, incorporating three-band model would require significantly more resources, particularly in calculating Chern numbers, which are numerically intensive for larger Hamiltonian matrices, this significantly cause a time consumption. For example, to achived the colored butterfly, it costs us around two days for a better resolution.

For other simple applications theoretical framework, we can intergrate with the \ac{NN} model, but in some more complex model it is crucial to consider the \ac{TNN} the implications of these findings in the context.
