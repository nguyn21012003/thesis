\chapter{\textbf{INTRODUCTION}}
Since the isolation of graphene in 2004, research on the field of \ac{2D} materials has noticeably grown into a major branch of physical science with a wide range of applications. Their unique \ac{2D} structure offers an open canvas to tailor and functionalize through layer number, defects, morphology, moir\'e patterns, strain, and other tunable properties. \ac{2D} materials, such as graphene and monolayer \ac{TMD}, have become a focal point in condensed matter physics due to their extraordinary electronic properties. Over the years, researchers have focused on graphene, with hope that its potential to replace silicon, enabling the continuation of Moore's law in next-generation semiconductor devices. However, in recent years, owing to their potentials, the number of scholarly papers researching graphene's use was enormous, reached a peak of thousand publications annually in 2021. Meanwhile, \ac{TMD} have created tremendous interest among materials scientists, on account of their huge potential for new types of electronics, optoelectronics and superconductivity. Atomically thin \ac{2D} \ac{TMD} have led to a variety of promising technologies for nanoelectronic, sensing and optoelectronics. Unlike graphene, many \ac{2D} \ac{TMD} are naturally semiconductors and enormously posses potential to be made into ultra-small and low power transistors than silicon. These materials consist of large atom from the transistion metal elements that lies in the middle of the periodic table are sandwiched by two layers of atoms from the chalcogenide elements, such as sulfur or selenium, forming a three-layer sandwich called a transistion metal dichalcogenides.

Another majority opportunity is unraveling the fundamental behind the unique properties of \ac{2D} materials. These systems exhibit a various interesting quantum effects, such as superconductivity, weak localization, tological insulation, and others. In some cases, these quantum effects can lead to novel applications, such as valleytronics or twistronics.

The Hofstadter butterfly have been studied for nearly 50 years with various ranging reportly from \ac{2D} symmetry lattice such as square lattice, triangular lattice, honeycomb lattice, Kagome lattice to Lieb's lattice, Mielke's lattice, Tasaki's lattice under an uniform magnetic field. The first Hofstadter butterfly was originally made for an electron moving in a \ac{2D} square lattice by computer scientist Douglas Hofstadter. The behaviour of an electron in such a quantum system could produce an energy spectrum which leads to a self-similar recursive Landau level spectrum resembling butterfly swings. The idea behind Hofstadter’s butterfly is that you’re looking at how the band structure of electrons moves when you have the magnetic field on one axis and the electrons’ energies on the other, and plotted on that diagram, the band forms a fractal structure that looks like a butterfly.

This thesis is mainly organized to explore the electronic structure of \ac{2D} material systems under external magnetic fields, with a particular focus on monolayer \ac{TMD}. 

We begin in Chapter~2 by presenting the theoretical background, including the tight-binding model, the crystal structure of \acp{TMD}, Landau quantization, and the Hall effect.

In Chapter~3, we study the three-band tight-binding model introduced in Ref.~\cite{PhysRevB.88.085433}, and then extend it by incorporating the effects of an external magnetic field.

Chapter~4 presents the results obtained from the model developed in Chapter~3. A portion of these results has been previously reported by other research groups Refs.~\cite{ho2014,ho2015}, but we also include new findings and a more general approach that broadens the scope of the original model. A key novelty of this work lies in the integration of theoretical elements from Chapter~2, such as Hall effects. To the best of our knowledge, such an explicit incorporation has not been addressed in the existing literature. This theoretical refinement enables a more systematic and physically grounded analysis of the Hofstadter spectrum in monolayer \acp{TMD}.



% Based on the minimal model introduced in Ref. \cite{PhysRevB.88.085433} we derived the tight-binding Hamiltonian without magnetic field and computed the bandstructure for \ac{TMD} monolayers. In Section 2.2, we first meets the Hofstadter butterfly of monolayer \ac{TMD} by deriving the Hamiltonian in the tight-binding model under a fractional magnetic field and analyzing the result Hofstadter spectrum. Delving into the Hofstadter physics, we begin exploring the butterfly with quantum phenomena such as Landau levels and quantum Hall effects. These are disscused in Section. 2.2 and Section 2.3. The results obtained exhibits an interesting phenomana. Ultimately, chapter 3 concludes the thesis by summarizing the focus key finding and highlighting the potential avenues for future research. This includes the exploration of interaction effects beyond tight-binding theory and the extension of material physics studies to other tunable properties, such as Moir\'e systems or twisted bilayer systems to \ac{2D} materials.
